\documentclass[18pt]{beamer}
\usepackage[utf8]{inputenc} % for the umlauts
\usepackage{subfigure}

\beamertemplatenavigationsymbolsempty
%% SLIDE FORMAT

% use 'beamerthemekit' for standard 4:3 ratio
% for widescreen slides (16:9), use 'beamerthemekitwide'

\usepackage{templates/beamerthemekit}
% \usepackage{templates/beamerthemekitwide}

\setcounter{tocdepth}{1}

%% TITLE PICTURE

% if a custom picture is to be used on the title page, copy it into the 'logos'
% directory, in the line below, replace 'mypicture' with the 
% filename (without extension) and uncomment the following line
% (picture proportions: 63 : 20 for standard, 169 : 40 for wide
% *.eps format if you use latex+dvips+ps2pdf, 
% *.jpg/*.png/*.pdf if you use pdflatex)

%\titleimage{mypicture}

%% TikZ INTEGRATION

% use these packages for PCM symbols and UML classes
% \usepackage{templates/tikzkit}
% \usepackage{templates/tikzuml}

% the presentation starts here

\usepackage{mathabx}
\usepackage{picture}
\usepackage[absolute,overlay]{textpos}
%\usepackage[texcoord,grid,gridunit=mm,gridcolor=red, subgridcolor=green]{eso-pic}
\setbeamercovered{invisible}
\setbeamertemplate{caption}{\raggedright\insertcaption\par}

\title[SWT1]{Softwaretechnik 1 - 3. Tutorium}
\subtitle{Tutorium 18}
\author{Felix Bachmann}
\date{05.06.2018}

\institute{KIT - Institut für Programmstrukturen und Datenorganisation (IPD)}

% Bibliography

\usepackage[citestyle=authoryear,bibstyle=numeric,hyperref,backend=biber]{biblatex}
\addbibresource{templates/example.bib}
\bibhang1em

\begin{document}

% change the following line to "ngerman" for German style date and logos
\selectlanguage{ngerman}

%title page
\begin{frame}
\titlepage
\end{frame}

\section{Orga}
	\subsection{Feedback 3. Übungsblatt}
	\begin{frame}
		\frametitle{3. Übungsblatt Statistik}
		\includegraphics[scale=0.7]{./pics/tut3/statistics-ub3.png}
		\linebreak \centering $\diameter$ 11,3 bzw. 15,7 von 27+3
	\end{frame}
	
	\subsection{3. Übungsblatt - Fehler}
	\begin{frame}
		\frametitle{Häufige Fehler}
		\begin{block}{Aufgabe 1 (Plug-In-Architektur: PluginManager + JmjrstPlugin)}
			\begin{itemize}
				\pause
				\item Plugins anhand \textbf{Klassen}namen vergleichen, nicht getName() \pause
				\item Strings auslagern (Konstanten oder Datei) \pause
				\item PluginManager gibt euch Iterable $\implies$ nutzt Iterator
				\begin{itemize}
					\item kein Casten, Kopieren in Liste \pause
				\end{itemize}
 				\item orientiert euch nicht am JMJRST-Stil
			\end{itemize}
		\end{block}
	\end{frame}

	\begin{frame}
		\frametitle{Häufige Fehler}
		\begin{block}{Aufgabe 2 (Plug-In)}
			\begin{itemize}
				\pause
				\item Prüfen auf *.png/*.jpg sollte case insensitive sein \pause
				\item Anm.: MetainfServices tut manchmal nicht richtig (oft hilft \texttt{mvn clean package}) \pause
			\end{itemize}
		\end{block}
		\begin{block}{Aufgabe 3 (iMage-Bundle)}
			\begin{itemize}
				\item keine :D
			\end{itemize}
		\end{block}
	\end{frame}

	\begin{frame}
		\frametitle{Häufige Fehler}
		\begin{block}{Aufgabe 4 (Aktivitätsdiagramm)}
			\begin{itemize}
				\pause 
				\item keine Partition verwendet \pause
				\item Aktivitiäten = runde Ecken, Objekte = spitze Ecken \pause
				\item denkt an die Rauten! \pause
				\item $\lbrack$Bedingung$\rbrack$ \pause
				\item verschachtelte Aktivitäten $\implies$ irgendwo passender Kasten dazu
			\end{itemize}
		\end{block}
	\end{frame}

	\begin{frame}
		\frametitle{Häufige Fehler}
		\begin{block}{Aufgabe 5 (Zustandsdiagramm)}
			\begin{itemize}
				\pause
				\item Übergänge,Zustände vergessen \pause
				\item Notation parallel: DxG \pause
				\item komplettes Diagramm hinzeichnen für Äquivalenz \pause
			\end{itemize}
		\end{block}
	\end{frame}

	\begin{frame}
		\frametitle{Häufige Fehler}
		\begin{block}{Aufgabe 6 (Sequenzdiagramm)}
			\begin{itemize}
				\pause 
				\item bzgl. Konstruktor sind VL-Folien etwas blöd \pause
				\item asynchron vs. synchron (Pfeilspitzen sind wichtig!) \pause
				\item nicht statische Instanzen unterstreichen \pause
				\item Instanz-Kästen erst dann hinzeichnen, wenn Instanz auch existiert \pause
				\item Selbstaufruf-Syntax
			\end{itemize}
		\end{block}
		\begin{block}{Aufgabe 7 (Testen mit Nachahmungen)}
			\begin{itemize}
				\pause
				\item Substitutionsprinzip: fordert dass Objekte einer Unterklasse immer auch im Kontext der Oberklasse eingesetzt werden können \pause
				\item Varianz war kein Problem, da Signatur gleich \pause
				\item Problem war Verhalten, schwächere Nachbedingung \pause
			\end{itemize}
		\end{block}
	\end{frame}



\section{Motivation}
	\subsection{Kontext}
		\begin{frame}
			\frametitle{Wo sind wir?}
			\begin{itemize}
				\item die ersten 2 Phasen des Wasserfallmodells sind geschafft
				\pause
				\linebreak $\implies$ Welche waren das nochmal? \pause Planung, Definition!
				\pause
				\linebreak $\implies$ Dokumente? \pause Lastenheft, Pflichtenheft (+ andere\dots)
				\pause
				\item  jetzt: Entwurf!
			\end{itemize}
		\end{frame}
	
		\begin{frame}
			\frametitle{Wozu Entwurf?}
			\centering
			\includegraphics[scale=0.4]{./pics/tut3/design.png} \linebreak
			Softwarearchitektur ist Grundlage für Implementierung!
		\end{frame}
	
		\begin{frame}
			\frametitle{Abgrenzung Definition vs. Entwurf}
			\begin{itemize}
				\item Definition: \textbf{Was} ist zu implementieren?
				\pause
				\item Entwurf: \textbf{Wie} ist das System zu implementieren?
			\end{itemize}
		\end{frame}
	
\section{Entwurfsmuster}
	\subsection{Grundlagen}
	
	\begin{frame}
		\frametitle{Empfehlenswerte Literatur (wirklich!)}
		knapp 700 Seiten \linebreak $\implies$ als interaktives Nachschlagewerk, falls man bestimmte Muster nicht versteht \linebreak
		\centering
		\includegraphics[scale=0.15]{./pics/tut3/literature.jpg}
	\end{frame}
		
	\begin{frame}
		\frametitle{Was sind Entwurfsmuster?}
		\begin{block}{Entwurfsmuster}
			Ein Software-Entwurfsmuster beschreibt eine
			Familie von Lösungen für ein Software-Entwurfsproblem.
		\end{block}
		\pause
		\begin{itemize}
			\item schematische Klassendiagramme zur Lösung von häufig auftretenden Problemen \pause
			\item Wiederverwendung von Entwurfswissen $\implies$ Rad nicht neu erfinden!
		\end{itemize}
		\pause
		\centering
		\includegraphics[scale=0.2]{./pics/tut3/new-wheel.jpg}
	\end{frame}

	\begin{frame}
		\frametitle{Wozu Entwurfsmuster?}
		\begin{itemize}
			\item erleichtern Kommunikation \pause
			\item erleichtern "'gute"' Entwürfe und das Schreiben von wartbarem/erweiterbarem Code
		\end{itemize}
\end{frame}
	
	\begin{frame}
		\frametitle{Geheimnisprinzip}
		\begin{block}{Geheimnis- / 
				Kapselungsprinzip}
			Jedes Modul verbirgt eine wichtige
			Entwurfsentscheidung hinter einer
			wohldefinierten Schnittstelle, die sich bei einer
			Änderung der Entscheidung nicht mit ändert.
		\end{block}
		\pause
		\begin{alertblock}{Warum eigentlich?}
			lokale Änderungen sollen sich nicht auf andere Teile auswirken 
			\linebreak $\implies$ weniger Fehler und Arbeit
		\end{alertblock}
		Beispiel? \pause $\implies$ private Attribute mit get()- und set()-Methoden
	\end{frame}

	\begin{frame}
		\frametitle{Vorgriff: Entwurfsmuster Strategie}
		\begin{itemize}
			\item Ziel: Algorithmen kapseln, austauschbar machen
			\item wird in vielen Entwurfsmustern verwendet
		\end{itemize}
		\includegraphics[scale=0.5]{./pics/tut3/strat.png}
	\end{frame}

	\begin{frame}
		\frametitle{Quiz (Ankreuzaufgaben aus Klausuren)}
		Wahr oder falsch?
		\begin{itemize}
			\item Das Entwurfsmuster Strategie bietet die Möglichkeit, eine Klasse mit einer von mehreren möglichen Verhaltensweisen zu konfigurieren. \pause \colorbox{green}{wahr} \pause
			\item Das Strategiemuster erfüllt das Geheimnisprinzip, indem es Datenstrukturen, die in einer konkreten Strategie enthalten sind, vor dem Klienten verbirgt. \pause \colorbox{green}{wahr} \pause
			\item Das Muster Strategie kapselt austauschbares Verhalten und verwendet Delegierung, um zu entscheiden, welches Verhalten verwendet wird. \pause \colorbox{green}{wahr} \pause 
			\item Das Hinzufügen einer neuen konkreten Strategie erfordert keine Änderung existierender konkreter Strategien. \pause \colorbox{green}{wahr}
		\end{itemize}
\end{frame}

	\begin{frame}
		\frametitle{Kategorien der Entwurfsmuster}
		\begin{itemize}
			\item \textbf{Entkopplungs-Muster}
				\begin{itemize}
					\item \textbf{Adapter}
					\item \textbf{Beobachter}
					\item \textbf{Iterator}
					\item \textbf{Stellvertreter}
					\item \textbf{Vermittler}
					\item Brücke
				\end{itemize}
			\item Varianten-Muster
			\item Zustandshandhabungs-Muster
			\item Steuerungs-Muster
			\item Bequemlichkeits-Muster
		\end{itemize}
	\end{frame}

	\begin{frame}
		\frametitle{Entkopplungs-Muster}
		\begin{itemize}
			\item übergeordnetes Ziel: System in Teile aufspalten, die unabhängig voneinander sind
			\linebreak $\implies$ Teile austauschbar bzw. veränderbar
	\end{itemize}
	\end{frame}

\section{Adapter}
	\subsection{Adapter}
	\begin{frame}
		\frametitle{Adapter}
		\begin{block}{Problem}
			\begin{itemize}
				\item Klassen mit inkompatiblen Schnittstellen, die wir aber zusammen benutzen wollen 
				\item Schnittstellen nicht änderbar (z.B. externe Bibliotheken)
			\end{itemize}
		\end{block}
		\pause
		\includegraphics[scale=0.45]{./pics/tut3/adap-obj.png}
	\end{frame}

	\begin{frame}
		\frametitle{Adapter (Objektadapter)}
		\includegraphics[scale=0.45]{./pics/tut3/adap-obj.png}
		\begin{block}{Wir sind bei Entkopplung-Mustern, Preisfrage:}
			Wo ist hier die Entkopplung?
			\pause
			\linebreak der Klient ist von der adaptierten Klasse entkoppelt $\implies$ austauschbar
		\end{block}
	\end{frame}

	\begin{frame}
		\frametitle{Adapter - Alternative (Klassenadapter)}
		\includegraphics[scale=0.45]{./pics/tut3/adap-cl.png} \linebreak \pause
		Was für ein Problem bekommt ihr, wenn ihr das auf einem ÜB implementieren müsst? \pause \linebreak
		$\implies$ keine Mehrfachvererbung in Java!
	\end{frame}

\section{Beobachter}
	\subsection{Beobachter}
	\begin{frame}{Beobachter/Observer: abstrakt}
		\begin{block}{Problem}
			\begin{itemize}
				\item ein Subjekt, viele Beobachter
				\item Subjekt ändert Zustand $\implies$ Beobachter machen "'irgendwas"
			\end{itemize}		
		\end{block}
	\end{frame}

	\begin{frame}{}
		\includegraphics[keepaspectratio, width=\textwidth, height=\textheight]{pics/tut3/obs.png}
		\pause
		\begin{block}{Entkopplung?}
		\begin{itemize}
			\pause 
			\item jeder Beobachter definiert, was bei Benachrichtigung passiert, Subjekt kriegt davon nichts mit \pause
			\item zur Laufzeit änderbar: Anzahl der Beobachter
		\end{itemize}
		\end{block}
	\end{frame}

	\begin{frame}{Beobachter/Observer: am Beispiel}
		\includegraphics[keepaspectratio, width=\textwidth, height=\textheight]{pics/tut3/observer_example.jpg}
	\end{frame}

\section{Iterator}
	\subsection{Iterator}
	\begin{frame}
		\frametitle{Iterator}
		\begin{block}{Problem}
			\begin{itemize}
				\item wollen über Datenstruktur iterieren + Operationen ausführen \linebreak $\implies$ Hinzufügen, Löschen\dots \pause
				\item das Ganze ohne Kentniss des internen Aufbaus der Datenstruktur 
			\end{itemize}
		\end{block}
		\pause
		\centering
		\includegraphics[scale=0.35]{./pics/tut3/iter.png}
	\end{frame}

	\begin{frame}
		\frametitle{Iterator}
		\centering
		\includegraphics[scale=0.35]{./pics/tut3/iter.png}
		\begin{block}{Entkopplung?}
			\begin{itemize}
				\pause 
				\item Klient benutzt nur Methoden der Schnittstelle auf dem konkreten Iterator \linebreak $\implies$ Implementierung austauschbar
			\end{itemize}
		\end{block}
	\end{frame}
	
	\begin{frame}
		\frametitle{Iterator}
		\centering
		\includegraphics[scale=0.35]{./pics/tut3/iter.png}
		\linebreak Beispiel in Java: list.iterator();
	\end{frame}

	\begin{frame}
		\frametitle{Quiz (Ankreuzaufgaben aus Klausuren)}
		Wahr oder falsch?
		\begin{itemize}
			\item Klienten können mithilfe des Iterator-Musters Sammlungen von Objekten und einzelne Objekte auf die gleiche Weise behandeln. \pause \colorbox{red}{falsch} \pause
			\item Das Entwurfsmuster Iterator ist den Variantenmustern zuzuordnen. \pause \colorbox{red}{falsch} 
		\end{itemize}
	\end{frame}

\section{Stellvertreter}
	\subsection{Stellvertreter}
	\begin{frame}
		\frametitle{Stellvertreter}
		\begin{block}{Problem}
			\begin{itemize}
				\item wollen Zugriff auf ein Objekt kontrollieren, ohne seine Klasse zu ändern \linebreak \pause $\implies$ Stellvertreter macht Zugriffskontrolle
			\end{itemize}
		\end{block}
		\pause
		\centering
		\includegraphics[scale=0.4]{./pics/tut3/prox.png}
	\end{frame}

	\begin{frame}
		\frametitle{Stellvertreter}
		\centering
		\includegraphics[scale=0.4]{./pics/tut3/prox.png}
		\begin{block}{Entkopplung?}
			\begin{itemize}
				\pause 
				\item Klient hat keinen direkten Zugriff auf das echte Subjekt
			\end{itemize}
		\end{block}
	\end{frame}

\section{Vermittler}
	\subsection{Vermittler}
		\begin{frame}
		\frametitle{Vermittler}
		\begin{block}{Problem}
			\begin{itemize}
				\item mehrere voneinander abhängige Objekte \linebreak \pause $\implies$ Zustände der Objekte von anderen Zuständen abhängig
			\end{itemize}
		\end{block}
		\pause
		\centering
		\includegraphics[scale=0.45]{./pics/tut3/med.png}
	\end{frame}

	\begin{frame}
		\frametitle{Vermittler}
		\centering
		\includegraphics[scale=0.45]{./pics/tut3/med.png}
		\begin{block}{Entkopplung?}
			\begin{itemize}
				\pause 
				\item Kollegen kennen sich nicht direkt  \linebreak \pause $\implies$ Hinzufügen eines Kollegen erfordert keine Änderung der alten Kollegen
			\end{itemize}
		\end{block}
	\end{frame}

\section{Klausuraufgabe}
	\subsection{Aufgabe}
	\begin{frame}
		\frametitle{Klausuraufgabe (Hauptklausur SS 2012)}
		\includegraphics[scale=0.35]{./pics/tut3/obs-task.png}	
		\begin{block}{Aufgabe 1}
			Welches Entwurfsmuster erkennen Sie in diesem Diagramm? \pause
			Beobachter.
		\end{block}
	\end{frame}

	\begin{frame}
			\begin{small}
				Entwerfen Sie das folgende Klassendiagramm passend zu dem Sequenzdiagramm; es soll
				alle verwendeten Klassen und Methoden enthalten. Kennzeichnen Sie die Zugreifbarkeiten
				der Methoden mit den Symbolen +, -, \#; seien Sie dabei möglichst restriktiv. Verzichten
				Sie auf die Modellierung von Attributen. Kennzeichnen Sie die Elemente
				des Entwurfsmusters und deren Funktion.
			\end{small}
			\linebreak
			\includegraphics[scale=0.35]{./pics/tut3/obs-task.png}
	\end{frame}

	\begin{frame}
		\frametitle{Musterlösung}
		\includegraphics[scale=0.35]{./pics/tut3/obs-task-sol.png}
	\end{frame}

%TODO java swing?

\section{Tipps}
	\subsection{Tipps}
	\begin{frame}
		\frametitle{Tipps - 4. Übungsblatt}
			\begin{exampleblock}{Aufgabe 1: iMage-GUI}
				\begin{itemize}
					\item macht die "'kleinen"' Bonusaufgaben 
					\linebreak $\implies$ relativ leichte Punkte \pause
					\item schaut euch die verschiedenen LayoutManager aus Java Swing an
					\linebreak $\implies$ verschiedene LayoutManager möglich (via mehrerer Container, z.B. JPanel)
				\end{itemize}
			\end{exampleblock}
			
			\pause
			
			\begin{exampleblock}{Aufgabe 2: Zustandsdiagramm (LEZ)}
				\begin{itemize}
					\item nochmal Syntax anschauen 
					\linebreak $\implies$ Was darf in Zustandsdiagramm, was nicht? (laut VL)
					\item von Hand!
				\end{itemize}
			\end{exampleblock}
	\end{frame}

	\begin{frame}
		\frametitle{Tipps - 4. Übungsblatt}
			\begin{exampleblock}{Aufgabe 3: Git}
				\begin{itemize}
					\item echo "hallo" $>>$ test.txt schreibt hallo in test.txt
					\item git-Dokumentation anschauen
				\end{itemize}
			\end{exampleblock}
			
			\pause

			\begin{exampleblock}{Aufgabe 4: Architekturstile}
				\begin{itemize}
					\item Jmjrst (gedanklich) umbauen
					\item Zusammenhang der Klassen anschauen (z.B. Main-Klasse)
				\end{itemize}
			\end{exampleblock}
	\end{frame}
	
	\subsection{Abgabe}
	\begin{frame}
		\frametitle{Denkt dran!}
		\begin{alertblock}{Abgabe}
			\begin{itemize}
				\item Deadline am 13.6 um 12:00
				\item A2-4 handschriftlich!
			\end{itemize}
		\end{alertblock}
	\end{frame}
		
	\begin{frame}
		\frametitle{Bis dann! (dann  := 19.06.17)}
		\centering
		\includegraphics[scale=0.5]{./comics/geek_and_poke_undeadCode.jpg}
	\end{frame}

\end{document}