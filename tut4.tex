\documentclass[18pt]{beamer}
\usepackage[utf8]{inputenc} % for the umlauts
\usepackage{subfigure}

\beamertemplatenavigationsymbolsempty
%% SLIDE FORMAT

% use 'beamerthemekit' for standard 4:3 ratio
% for widescreen slides (16:9), use 'beamerthemekitwide'

\usepackage{templates/beamerthemekit}
% \usepackage{templates/beamerthemekitwide}

\setcounter{tocdepth}{1}

%% TITLE PICTURE

% if a custom picture is to be used on the title page, copy it into the 'logos'
% directory, in the line below, replace 'mypicture' with the 
% filename (without extension) and uncomment the following line
% (picture proportions: 63 : 20 for standard, 169 : 40 for wide
% *.eps format if you use latex+dvips+ps2pdf, 
% *.jpg/*.png/*.pdf if you use pdflatex)

%\titleimage{mypicture}

%% TikZ INTEGRATION

% use these packages for PCM symbols and UML classes
% \usepackage{templates/tikzkit}
% \usepackage{templates/tikzuml}

% the presentation starts here

\usepackage{mathabx}
\usepackage{picture}
\usepackage[absolute,overlay]{textpos}
%\usepackage[texcoord,grid,gridunit=mm,gridcolor=red, subgridcolor=green]{eso-pic}
\setbeamercovered{invisible}
\setbeamertemplate{caption}{\raggedright\insertcaption\par}

\title[SWT1]{Softwaretechnik 1 - 4. Tutorium}
\subtitle{Tutorium 03}
\author{Felix Bachmann}
\date{26.06.2017}

\institute{KIT - Institut für Programmstrukturen und Datenorganisation (IPD)}

% Bibliography

\usepackage[citestyle=authoryear,bibstyle=numeric,hyperref,backend=biber]{biblatex}
\addbibresource{templates/example.bib}
\bibhang1em

\begin{document}

% change the following line to "ngerman" for German style date and logos
\selectlanguage{ngerman}

%title page
\begin{frame}
\titlepage
\end{frame}

\begin{frame}
\tableofcontents
\end{frame}


\section{Orga}

	%TODO add design patterns extra slides

	\subsection{Allgemein}
	\begin{frame}
		\frametitle{Allgemeines}
		\begin{alertblock}{Ansage der Übungsleiter}
			\begin{itemize}
				\item ab jetzt keine Abgabe per Mail mehr!
				\linebreak $\implies$ auch nicht in Ausnahmefällen
			\end{itemize}
		\end{alertblock}
	\end{frame}

	\subsection{Feedback 4. Übungsblatt}
	\begin{frame}
		\frametitle{4. Übungsblatt Statistik}
		%TODO statistics\includegraphics[scale=0.7]{./pics/tut3/statistics-ub3.png}
		%TODO avg \linebreak \centering $\diameter$ 
	\end{frame}
	
	\subsection{4. Übungsblatt - Fehler (Allgemein)}
	\begin{frame}
		\frametitle{Häufige Fehler}
		\begin{block}{Allgemein}
			\begin{itemize}
				\item %TODO 
			\end{itemize}
		\end{block}
	\end{frame}
	
	\subsection{4. Übungsblatt - Fehler}
	\begin{frame}
		\frametitle{Häufige Fehler}
		\begin{block}{Aufgabe 1 (Zustandsdiagramm - LEZ): $\diameter$}%TODO avg
			\begin{itemize}
				\item %TODO
			\end{itemize}
		\end{block}
	\end{frame}

	\begin{frame}
		\frametitle{Häufige Fehler}
		\begin{block}{Aufgabe 2 (Abbottsche Methode): $\diameter$}%TODO avg
			\begin{itemize}
				\item %TODO
			\end{itemize}
		\end{block}
		\begin{block}{Aufgabe 3 (iMage-GUI): $\diameter$}%TODO avg
			\begin{itemize}
				\item	%TODO
			\end{itemize}
		\end{block}
	\end{frame}

	\begin{frame}
		\frametitle{Häufige Fehler}
		\begin{block}{Aufgabe 4 (Geheimnisprinzip): $\diameter$}%TODO avg
			\begin{itemize}
				\item %TODO
			\end{itemize}
		\end{block}
	\end{frame}

\section{Recap}
	\subsection{Quiz(Adapter)}
	\begin{frame}
		\frametitle{Was bisher geschah..}
		\begin{itemize}
			\item haben uns erste Entkopplungmuster angeschaut \pause
			\linebreak $\implies$ Beobachter, Iterator, Adapter
		\end{itemize}
		\begin{figure}
			\includegraphics[scale=0.33]{./pics/tut4/adap-obj-mod.png}
		\end{figure}
		Welches Entwurfsmuster? \pause (Objekt-)Adapter
	\end{frame}
	
	\begin{frame}
		\frametitle{Was bisher geschah..}
		\begin{itemize}
			\item haben uns erste Entkopplungmuster angeschaut
			\linebreak $\implies$ Beobachter, Iterator, Adapter
		\end{itemize}
		\begin{figure}
			\includegraphics[scale=0.33]{./pics/tut4/adap-obj-mod.png}
		\end{figure}
		Welche Klassen?
	\end{frame}
	
	\begin{frame}
		\frametitle{Was bisher geschah..}
		\begin{itemize}
			\item haben uns erste Entkopplungmuster angeschaut
			\linebreak $\implies$ Beobachter, Iterator, Adapter
		\end{itemize}
		\begin{figure}
			\includegraphics[scale=0.45]{./pics/tut3/adap-obj.png}
		\end{figure}
	\end{frame}
	
	\subsection{Quiz (Iterator)}
	\begin{frame}
		\frametitle{Was bisher geschah..}
		\begin{itemize}
			\item haben uns erste Entkopplungmuster angeschaut
			\linebreak $\implies$ Beobachter, Iterator, Adapter
		\end{itemize}
		\begin{figure}
			\includegraphics[scale=0.25]{./pics/tut4/iter-mod.png}
		\end{figure}
		Welches Entwurfsmuster? \pause Iterator
	\end{frame}
	
	\begin{frame}
		\frametitle{Was bisher geschah..}
		\begin{itemize}
			\item haben uns erste Entkopplungmuster angeschaut
			\linebreak $\implies$ Beobachter, Iterator, Adapter
		\end{itemize}
		\begin{figure}
			\includegraphics[scale=0.25]{./pics/tut4/iter-mod.png}
		\end{figure}
		Welche Klassen und Methoden?
	\end{frame}
	
	\begin{frame}
		\frametitle{Was bisher geschah..}
		\begin{itemize}
			\item haben uns erste Entkopplungmuster angeschaut
			\linebreak $\implies$ Beobachter, Iterator, Adapter
		\end{itemize}
		\begin{figure}
			\includegraphics[scale=0.35]{./pics/tut3/iter.png}
		\end{figure}
	\end{frame}
	
	\subsection{Quiz(Beobachter)}
	
	\begin{frame}
		\frametitle{Was bisher geschah..}
		\begin{itemize}
			\item haben uns erste Entkopplungmuster angeschaut
			\linebreak $\implies$ Beobachter, Iterator, Adapter
		\end{itemize}
		\begin{figure}
			\includegraphics[scale=0.25]{./pics/tut4/obs-mod.png}
		\end{figure}
		\pause Ist wohl ein Beobachter :) \pause Klassen, Methoden?
	\end{frame}
	
	\begin{frame}
		\frametitle{Was bisher geschah..}
		\begin{itemize}
			\item haben uns erste Entkopplungmuster angeschaut
			\linebreak $\implies$ Beobachter, Iterator, Adapter
		\end{itemize}
		\begin{figure}
			\includegraphics[scale=0.35]{./pics/tut3/obs.png}
		\end{figure}
	\end{frame}

	\begin{frame}
		\frametitle{Kategorien der Entwurfsmuster}
		\begin{itemize}
			\item \textbf{Entkopplungs-Muster}
			\begin{itemize}
				\item Adapter \colorbox{green}{fertig}
				\item Beobachter\colorbox{green}{fertig}
				\item Iterator \colorbox{green}{fertig}
				\item \textbf{Stellvertreter}
				\item \textbf{Vermittler}
				\item (Brücke)
			\end{itemize}
			\item Varianten-Muster
			\item Zustandshandhabungs-Muster
			\item Steuerungs-Muster
			\item Bequemlichkeits-Muster
		\end{itemize}
	\end{frame}


\section{Stellvertreter}
	\subsection{Stellvertreter}
	\begin{frame}
		\frametitle{Stellvertreter}
		\begin{block}{Problem}
			\begin{itemize}
				\item wollen Zugriff auf ein Objekt kontrollieren, ohne seine Klasse zu ändern \linebreak \pause $\implies$ Stellvertreter macht Zugriffskontrolle
			\end{itemize}
		\end{block}
		\pause
		\centering
		\includegraphics[scale=0.4]{./pics/tut3/prox.png}
	\end{frame}

	\begin{frame}
		\frametitle{Stellvertreter}
		\centering
		\includegraphics[scale=0.4]{./pics/tut3/prox.png}
		\begin{block}{Entkopplung?}
			\begin{itemize}
				\pause 
				\item Klient hat keinen direkten Zugriff auf das echte Subjekt
				\pause 
				\item Stellvertreter hat Relation zu Oberklasse (!), echtes Subjekt austauschbar
			\end{itemize}
		\end{block}
	\end{frame}

\section{Vermittler}
	\subsection{Vermittler}
		\begin{frame}
		\frametitle{Vermittler}
		\begin{block}{Problem}
			\begin{itemize}
				\item mehrere voneinander abhängige Objekte \linebreak \pause $\implies$ Zustände der Objekte von anderen Zuständen abhängig
			\end{itemize}
		\end{block}
		\pause
		\centering
		\includegraphics[scale=0.45]{./pics/tut3/med.png}
	\end{frame}

	\begin{frame}
		\frametitle{Vermittler}
		\centering
		\includegraphics[scale=0.45]{./pics/tut3/med.png}
		\begin{block}{Entkopplung?}
			\begin{itemize}
				\pause 
				\item Kollegen kennen sich nicht direkt  \linebreak \pause $\implies$ Hinzufügen eines Kollegen erfordert keine Änderung der alten Kollegen
			\end{itemize}
		\end{block}
	\end{frame}

\section{Gruppenarbeit}
\subsection{Gruppenarbeit}
	\begin{frame}
		\frametitle{Kategorien der Entwurfsmuster}
		\begin{itemize}
			\item Entkopplungs-Muster \colorbox{green}{fertig}
			\item \textbf{Varianten-Muster}
			\begin{itemize}
				\item (Abstrakte Fabrik)
				\item (Besucher)
				\item Schablonenmethode
				\item Fabrikmethode
				\item Kompositum
				\item Strategie \colorbox{green}{fertig}
				\item Dekorierer
			\end{itemize}
			\item Zustandshandhabungs-Muster
			\item Steuerungs-Muster
			\item Bequemlichkeits-Muster
		\end{itemize}
	\end{frame}
	
	\begin{frame}
		\frametitle{Varianten-Muster}
		\begin{block}{Übergeordnetes Ziel}
			\begin{itemize}
				\item übergeordnetes Ziel: Gemeinsamkeiten herausziehen und an einer Stelle beschreiben \pause
				\linebreak $\implies$ keine Wiederholung desselben Codes \pause
				\linebreak $\implies$ bessere Wartbarkeit/Erweiterbarkeit		
			\end{itemize}
		\end{block}
	\end{frame}
	
	\begin{frame}
		\frametitle{Jetzt: Gruppenarbeit}
		\begin{enumerate}
			\item ihr kriegt pro Reihe eine Aufgabe
			\item ihr habt Zeit zum Bearbeiten
			\item ihr stellt den anderen eure Lösung vor
		\end{enumerate}
	\end{frame}

	\begin{frame}
		\frametitle{Vorstellung Dekorierer}
		\includegraphics[scale=0.35]{./pics/tut4/decor.png}
	\end{frame}

	\begin{frame}
		\frametitle{MuLö Dekorierer}
		\begin{block}{Wo Gemeinsamkeiten?}
			Die beiden Methoden methodeEins() und methodeZwei().
		\end{block}
		\begin{block}{Wo Variation?}
			In den KonkretenDekorierern bzw. ihren Methoden. Hier: neueMethodeA(), neueMethodeB().
		\end{block}
		\begin{block}{Wozu Instanzvariable?}
			Weiterleitung von Aufrufen der methodeEins() und methodeZwei() an die KonkreteKompenente.
		\end{block}
	\end{frame}

	\begin{frame}
		\frametitle{Vorstellung Kompositum}
		\includegraphics[scale=0.35]{./pics/tut4/comp.png}
	\end{frame}

	\begin{frame}
		\frametitle{MuLö Kompositum}
		\begin{block}{Wo Gemeinsamkeiten?}
			gemeinsameOperation().
		\end{block}
		\begin{block}{Wo Variation?}
			In Blatt/Kompositum-Klassen mit verschiedenen zusätzlichen Operationen.
		\end{block}
		\begin{block}{Zusammengesetzt vs. nicht-zusammengesetzt}
			Kompositum = zusammengesetzt, Blatt = nicht-zusammengesetzt
		\end{block}
	\end{frame}

	\begin{frame}
		\frametitle{Vorstellung Schablonenmethode}
		\includegraphics[scale=0.45]{./pics/tut4/schab.png}
	\end{frame}

	\begin{frame}
		\frametitle{MuLö Schablonenmethode}
		\begin{block}{Wo Gemeinsamkeiten?}
			Reihenfolge der Methodenaufrufe in der Schablonenmethode.
		\end{block}
		\begin{block}{Wo Variation?}
			In den Einschubmethoden. (hier: primitiveOperation1() und primitiveOpoeration2())
		\end{block}
		\begin{block}{Schablonenmethode vs. Einschubmethode}
			Einschubmethode ist eine der Methoden, die von der Schablonenmethode aufgerufen wird und deren Implementierung in den Unterklassen stattfindet.
		\end{block}
	\end{frame}

	\begin{frame}
		\frametitle{Vorstellung Fabrikmethode}
		\includegraphics[scale=0.4]{./pics/tut4/fab.png}
	\end{frame}

	\begin{frame}
		\frametitle{MuLö Fabrikmethode}
		\begin{block}{Wo Gemeinsamkeiten?}
			Reihenfolge der Methodenaufrufe in der beliebigenMethode().
		\end{block}
		\begin{block}{Wo Variation?}
			In der Fabrikmethode.
		\end{block}
		\begin{block}{Klasse des Objekts, Oberklasse, Unterklasse}
			Klasse des Objekts = KonkretesProdukt, Oberklasse = Produkt, Unterklasse = KonkreterErzeuger
		\end{block}
		\begin{block}{Unterschied zu Schablonenmethode?}
			Fabrikmethode benutzen, wenn ein Objekt erzeugt wird. Fabrikmethode ist Einschubmethode des Musters "'Schablonenmethode"'.
		\end{block}
		\begin{block}{Wahr/falsch}
			Fabrikmethode ist eine Einschubmethode, keine Schablonenmethode.
		\end{block}
	\end{frame}

\section{Memento}
	\subsection{Intro}
	\begin{frame}
		\frametitle{Kategorien der Entwurfsmuster}
		\begin{itemize}
			\item Entkopplungs-Muster \colorbox{green}{fertig}
			\item Varianten-Muster \colorbox{green}{fertig}
			\item \textbf{Zustandshandhabungs-Muster}
				\begin{itemize}
					\item (Einzelstück)
					\item (Fliegengewicht)
					\item \textbf{Memento} 
					\item (Prototyp) 
					\item (Zustand)
				\end{itemize}
			\item Steuerungs-Muster
			\item Bequemlichkeits-Muster
		\end{itemize}
	\end{frame}

	\begin{frame}
		\frametitle{Zustandshandhabungs-Muster}
		\begin{block}{Übergeordnetes Ziel}
			\begin{itemize}
				\item den Zustand eines Objektes beschreiben (wer hätt's gedacht? :D) \pause 
				\item aber unabhängig von dem Zweck des Objekts!
			\end{itemize}
		\end{block}
	\end{frame}

	\begin{frame}
		\frametitle{Memento}
		\begin{block}{Problem}
			\begin{itemize}
				\item internen Zustand eines Objekts "'externalisieren"', um z.B. Zurücksetzen möglich zu machen \pause 
				\item ohne Kapselung zu verletzten!
			\end{itemize}
		\end{block}
		\pause
		\centering
		\includegraphics[scale=0.4]{./pics/tut4/mem.png}
	\end{frame}

	\begin{frame}
		\frametitle{Memento}
		\includegraphics[scale=0.4]{./pics/tut4/mem.png}
		\begin{block}{Problem gelöst?}
			\begin{itemize}
				\pause
				\item Ja
				\begin{itemize}
					\pause
					\item Zustand durch Memento externalisiert \pause
					\item Kapselung nicht verletzt (Nutzer ruft nur sichereZustand() auf und kriegt neuen Memento)
				\end{itemize}
			\end{itemize}
		\end{block}
\end{frame}
	
\section{Tipps}
	\subsection{Tipps}
	\begin{frame}
		\frametitle{Tipps - 5. Übungsblatt}
			\begin{exampleblock}{Aufgabe 1: Manager-Deutsch und Architekturstile} 
				\begin{itemize}
					\item Architekturstile nochmal anschauen
				\end{itemize}
			\end{exampleblock}
			\pause
			\begin{exampleblock}{Aufgabe 2: Iterator für Plug-Ins} 
				\begin{itemize}
					\item Iterator-Muster selbst benutzen
				\end{itemize}
			\end{exampleblock}
	\end{frame}

	\begin{frame}
		\frametitle{Tipps - 5. Übungsblatt}
			\begin{exampleblock}{Aufgabe 3: Geometrify mit Entwurfsmustern}
				\begin{itemize}
					\item überlegen, welches Entwurfsmuster \textbf{warum} Sinn macht
				\end{itemize}
			\end{exampleblock}
			\pause
			\begin{exampleblock}{Aufgabe 4: Geometrify umstrukturieren}
				\begin{itemize}
					\item Überlegungen aus Aufgabe 3 umsetzen
				\end{itemize}
			\end{exampleblock}
	\end{frame}

	\begin{frame}
		\frametitle{Tipps - 5. Übungsblatt}
		\begin{exampleblock}{Aufgabe 5: GUI erweitern}
			\begin{itemize}
				\item nochmal ServiceLoader
			 	 $\implies$ diesmal mit Primitiven
			\end{itemize}
		\end{exampleblock}
	\end{frame}
	
	\subsection{Abgabe}
	\begin{frame}
		\frametitle{Denkt dran!}
		\begin{alertblock}{Abgabe}
			\begin{itemize}
				\item Deadline am 5.7. um 12:00
				\item Aufgabe 1, 3 handschriftlich
			\end{itemize}
		\end{alertblock}
	\end{frame}
		
	\begin{frame}
		\frametitle{Bis dann! (dann  := 10.07.17)}
		\centering
		\includegraphics[scale=0.4]{./comics/patterns.jpg}
	\end{frame}

\end{document}