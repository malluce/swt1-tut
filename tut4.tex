\documentclass[18pt]{beamer}
\usepackage[utf8]{inputenc} % for the umlauts
\usepackage{subfigure}

\beamertemplatenavigationsymbolsempty
%% SLIDE FORMAT

% use 'beamerthemekit' for standard 4:3 ratio
% for widescreen slides (16:9), use 'beamerthemekitwide'

\usepackage{templates/beamerthemekit}
% \usepackage{templates/beamerthemekitwide}

\setcounter{tocdepth}{1}

%% TITLE PICTURE

% if a custom picture is to be used on the title page, copy it into the 'logos'
% directory, in the line below, replace 'mypicture' with the 
% filename (without extension) and uncomment the following line
% (picture proportions: 63 : 20 for standard, 169 : 40 for wide
% *.eps format if you use latex+dvips+ps2pdf, 
% *.jpg/*.png/*.pdf if you use pdflatex)

%\titleimage{mypicture}

%% TikZ INTEGRATION

% use these packages for PCM symbols and UML classes
% \usepackage{templates/tikzkit}
% \usepackage{templates/tikzuml}

% the presentation starts here

\usepackage{csquotes}
\usepackage{mathabx}
\usepackage{picture}
\usepackage[absolute,overlay]{textpos}
%\usepackage[texcoord,grid,gridunit=mm,gridcolor=red, subgridcolor=green]{eso-pic}
\setbeamercovered{invisible}
\setbeamertemplate{caption}{\raggedright\insertcaption\par}

\title[SWT1]{Softwaretechnik 1 - 4. Tutorium}
\subtitle{Tutorium 18}
\author{Felix Bachmann}
\date{19.06.2018}

\institute{KIT - Institut für Programmstrukturen und Datenorganisation (IPD)}

% Bibliography

\begin{document}

% change the following line to "ngerman" for German style date and logos
\selectlanguage{ngerman}

%title page
\begin{frame}
\titlepage
\end{frame}


\section{Orga}
		\subsection{Feedback 3. Übungsblatt}
	\begin{frame}
	\frametitle{3. Übungsblatt Statistik}
	\includegraphics[scale=0.7]{./pics/tut3/statistics-ub3.png}
	\linebreak \centering $\diameter$ 11,3 bzw. 15,7 von 27+3
\end{frame}

\subsection{3. Übungsblatt - Fehler}
\begin{frame}
\frametitle{Häufige Fehler}
\begin{block}{Aufgabe 1 (Plug-In-Architektur: PluginManager + JmjrstPlugin)}
\begin{itemize}
	\pause
	\item Plugins anhand \textbf{Klassen}namen vergleichen, nicht getName() \pause
	\item Strings auslagern (Konstanten oder Datei) \pause
	\item PluginManager gibt euch Iterable $\implies$ nutzt Iterator
	\begin{itemize}
		\item kein Casten, Kopieren in Liste \pause
	\end{itemize}
	\item orientiert euch nicht am JMJRST-Stil
\end{itemize}
\end{block}
\end{frame}

\begin{frame}
\frametitle{Häufige Fehler}
\begin{block}{Aufgabe 2 (Plug-In)}
\begin{itemize}
\pause
\item Prüfen auf *.png/*.jpg sollte case insensitive sein \pause
\item Anm.: MetainfServices tut manchmal nicht richtig (oft hilft \texttt{mvn clean package}) \pause
\end{itemize}
\end{block}
\begin{block}{Aufgabe 3 (iMage-Bundle)}
\begin{itemize}
\item keine :D
\end{itemize}
\end{block}
\end{frame}

\begin{frame}
\frametitle{Häufige Fehler}
\begin{block}{Aufgabe 4 (Aktivitätsdiagramm)}
\begin{itemize}
\pause 
\item keine Partition verwendet \pause
\item Aktivitiäten = runde Ecken, Objekte = spitze Ecken \pause
\item denkt an die Rauten! \pause
\item $\lbrack$Bedingung$\rbrack$ \pause
\item verschachtelte Aktivitäten $\implies$ irgendwo passender Kasten dazu
\end{itemize}
\end{block}
\end{frame}

\begin{frame}
\frametitle{Häufige Fehler}
\begin{block}{Aufgabe 5 (Zustandsdiagramm)}
\begin{itemize}
\pause
\item Übergänge,Zustände vergessen \pause
\item Notation parallel: DxG \pause
\item komplettes Diagramm hinzeichnen für Äquivalenz \pause
\end{itemize}
\end{block}
\end{frame}

\begin{frame}
\frametitle{Häufige Fehler}
\begin{block}{Aufgabe 6 (Sequenzdiagramm)}
\begin{itemize}
\pause 
\item bzgl. Konstruktor sind VL-Folien etwas blöd \pause
\item asynchron vs. synchron (Pfeilspitzen sind wichtig!) \pause
\item nicht statische Instanzen unterstreichen \pause
\item Instanz-Kästen erst dann hinzeichnen, wenn Instanz auch existiert \pause
\item Selbstaufruf-Syntax
\end{itemize}
\end{block}
\begin{block}{Aufgabe 7 (Testen mit Nachahmungen)}
\begin{itemize}
\pause
\item Substitutionsprinzip: fordert dass Objekte einer Unterklasse immer auch im Kontext der Oberklasse eingesetzt werden können \pause
\item Varianz war kein Problem, da Signatur gleich \pause
\item Problem war Verhalten, schwächere Nachbedingung \pause
\end{itemize}
\end{block}
\end{frame}


	\subsection{Feedback 4. Übungsblatt}
	\begin{frame}
		\frametitle{4. Übungsblatt Statistik}
		\includegraphics[scale=0.7]{./pics/tut4/statistics-ub4.png}
		\linebreak \centering $\diameter$ 9,3 (alle), 14,5 (abgegeben) von 25+3
	\end{frame}
	
	\subsection{4. Übungsblatt - Fehler}
	\begin{frame}
		\frametitle{Häufige Fehler}
		\begin{block}{Aufgabe 1: GUI für iMage}
			\includegraphics[scale=0.34]{./pics/tut5/file-resource.png}
			\begin{itemize}
				\pause 
				\item Gottklassen, wir wollen aber sinnvolle Objektorientierung! \pause
				\item $\texttt{SwingUtilities.invokeLater(e -> startGui())}$ benutzen:
				\linebreak $\implies$ Thread-Safe (siehe nächstes Tut) \pause
			\end{itemize}
		\end{block}
	\end{frame}

	\begin{frame}
		\frametitle{Häufige Fehler}
		\begin{block}{Aufgabe 2: Zustandsdiagramm für Wasserzeichnen}
			\begin{itemize}
				\item $a() [b] \ne [b] /a()$ \linebreak $\implies$ beim skalieren/exception werfen
				\pause 
				\item sowohl \enquote{validiertes Bild} als auch \enquote{skaliertes Bild} ein Zustand
			\end{itemize}
		\end{block}
		\pause 
		\begin{block}{Aufgabe 3: git}
			\begin{itemize}
				\item bei Umbenennung und Änderung direkt zu commiten würde beides hinzufügen \pause
				\linebreak $\implies$ entweder \texttt{add -N}, \texttt{add -p} \pause
				\linebreak $\implies$ oder \texttt{mv neu alt}, \texttt{git mv alt neu} \pause
				\item git rm -r löscht rekursiv Ordner (inkl. der Überordner!)
				\begin{itemize}
					\item und fügt implizit zur Staging Area hinzu, kein add nötig
				\end{itemize}
			\end{itemize}
		\end{block}
	\end{frame}

	\begin{frame}
		\frametitle{Häufige Fehler}
		\begin{block}{Aufgabe 4: Architekturstile für JMJRST}
			\begin{itemize}
				\item Zuordnung begründen, wenn unklar \pause
				\item Main eindeutig zugeordnet \pause
				\item Änderungen zu vage beschrieben
			\end{itemize}
		\end{block}
	\end{frame}

	\begin{frame}
	\frametitle{Evaluation}
	leider nur 5 Teilnehmer??
	\centering
	\begin{table}
		\begin{tabular}{|c|c|}
			\hline 
			Gut & Schlecht/Verbesserungswürdig \\ 
			\hline
			Folien (3) & \\
			Beispiele und Code (3) & \\ 
			Erklärungen (2) & \\
			Tipps (2) & \\
			Aufgaben (2)&  zu viel Zeit für Aufgaben (2)\\ 
			&  Wahr/Falsch zu einfach (1)\\ 
			& nicht immer Lösungen auf Folie (1)\\
			& gleiche Beispiele wie in VL (1) \\
			& Bewertungen zu kurz (1)\\
			\hline 
		\end{tabular} 
	\end{table}

	\end{frame}

\section{Recap}
	\subsection{Quiz(Adapter)}
	\begin{frame}
		\frametitle{Was bisher geschah..}
		\begin{itemize}
			\item haben uns Entkopplungmuster angeschaut \pause
			\linebreak $\implies$ Beobachter, Iterator, Adapter, Stellvertreter \pause
		\end{itemize}
		\begin{figure}
			\includegraphics[scale=0.33]{./pics/tut4/adap-obj-mod.png}
		\end{figure}
		Welches Entwurfsmuster? \pause (Objekt-)Adapter
	\end{frame}
	
	\begin{frame}
		\frametitle{Was bisher geschah..}
		\begin{itemize}
			\item haben uns Entkopplungmuster angeschaut
			\linebreak $\implies$ Beobachter, Iterator, Adapter, Stellvertreter
		\end{itemize}
		\begin{figure}
			\includegraphics[scale=0.33]{./pics/tut4/adap-obj-mod.png}
		\end{figure}
		Welche Klassen?
	\end{frame}
	
	\begin{frame}
		\frametitle{Was bisher geschah..}
		\begin{itemize}
			\item haben uns Entkopplungmuster angeschaut
			\linebreak $\implies$ Beobachter, Iterator, Adapter, Stellvertreter
		\end{itemize}
		\begin{figure}
			\includegraphics[scale=0.45]{./pics/tut3/adap-obj.png}
		\end{figure}
	\end{frame}
	
	\subsection{Quiz (Iterator)}
	\begin{frame}
		\frametitle{Was bisher geschah..}
		\begin{itemize}
			\item haben uns Entkopplungmuster angeschaut
			\linebreak $\implies$ Beobachter, Iterator, Adapter, Stellvertreter
		\end{itemize}
		\begin{figure}
			\includegraphics[scale=0.25]{./pics/tut4/iter-mod.png}
		\end{figure}
		Welches Entwurfsmuster? \pause Iterator
	\end{frame}
	
	\begin{frame}
		\frametitle{Was bisher geschah..}
		\begin{itemize}
			\item haben uns Entkopplungmuster angeschaut
			\linebreak $\implies$ Beobachter, Iterator, Adapter, Stellvertreter
		\end{itemize}
		\begin{figure}
			\includegraphics[scale=0.25]{./pics/tut4/iter-mod.png}
		\end{figure}
		Welche Klassen und Methoden?
	\end{frame}
	
	\begin{frame}
		\frametitle{Was bisher geschah..}
		\begin{itemize}
			\item haben uns Entkopplungmuster angeschaut
			\linebreak $\implies$ Beobachter, Iterator, Adapter, Stellvertreter
		\end{itemize}
		\begin{figure}
			\includegraphics[scale=0.35]{./pics/tut3/iter.png}
		\end{figure}
	\end{frame}
	
	\subsection{Quiz(Beobachter)}
	
	\begin{frame}
		\frametitle{Was bisher geschah..}
		\begin{itemize}
			\item haben uns Entkopplungmuster angeschaut
			\linebreak $\implies$ Beobachter, Iterator, Adapter, Stellvertreter
		\end{itemize}
		\begin{figure}
			\includegraphics[scale=0.25]{./pics/tut4/obs-mod.png}
		\end{figure}
		\pause Ist wohl ein Beobachter :) \pause Methoden?
	\end{frame}
	
	\begin{frame}
		\frametitle{Was bisher geschah..}
		\begin{itemize}
			\item haben uns Entkopplungmuster angeschaut
			\linebreak $\implies$ Beobachter, Iterator, Adapter, Stellvertreter
		\end{itemize}
		\begin{figure}
			\includegraphics[scale=0.35]{./pics/tut3/obs.png}
		\end{figure}
	\end{frame}

\section{Vermittler}
\subsection{Letztes Entkopplungsmuster: Vermittler}
	\begin{frame}
	\frametitle{Vermittler}
	\begin{block}{Problem}
		\begin{itemize}
			\item mehrere voneinander abhängige Objekte \linebreak \pause $\implies$ Zustände der Objekte von anderen Zuständen abhängig
		\end{itemize}
	\end{block}
	\pause
	\centering
	\includegraphics[scale=0.45]{./pics/tut3/med.png}
\end{frame}

\begin{frame}
	\frametitle{Vermittler}
	\centering
	\includegraphics[scale=0.45]{./pics/tut3/med.png}
	\begin{block}{Entkopplung?}
	\begin{itemize}
		\pause 
		\item Kollegen kennen sich nicht direkt  \linebreak \pause $\implies$ Hinzufügen eines Kollegen erfordert keine Änderung der alten Kollegen
	\end{itemize}
	\end{block}
\end{frame}

\begin{frame}{Beobachter vs. Vermittler}
	\begin{itemize}
	\item wirken ähnlich
	\begin{itemize}
	\item ein Vermittler, viele Kollegen
	\item ein Subjekt, viele Beobachter
	\end{itemize}
	\end{itemize}
	\pause
	\begin{block}{Beobachter}
	Beobachter interessieren sich nicht füreinander. Nur für das Subjekt
	\end{block}
	\begin{block}{Vermittler}
	Kollegen interessieren sich füreinander, kommunizieren über den Vermittler miteinander.
	\end{block}
\end{frame}

\subsection{Aufgabe}
	\begin{frame}
	\frametitle{Klausuraufgabe (Hauptklausur SS 2012)}
	\includegraphics[scale=0.35]{./pics/tut3/obs-task.png}	
	\begin{block}{Aufgabe 1}
	Welches Entwurfsmuster erkennen Sie in diesem Diagramm? \pause
	Beobachter.
	\end{block}
\end{frame}

\begin{frame}
	\begin{small}
	Entwerfen Sie das folgende Klassendiagramm passend zu dem Sequenzdiagramm; es soll
	alle verwendeten Klassen und Methoden enthalten. Kennzeichnen Sie die Zugreifbarkeiten
	der Methoden mit den Symbolen +, -, \#; seien Sie dabei möglichst restriktiv. Verzichten
	Sie auf die Modellierung von Attributen. Kennzeichnen Sie die Elemente
	des Entwurfsmusters und deren Funktion.
	\end{small}
	\linebreak
	\includegraphics[scale=0.35]{./pics/tut3/obs-task.png}
\end{frame}

\begin{frame}
\frametitle{Musterlösung}
\includegraphics[scale=0.35]{./pics/tut3/obs-task-sol.png}
\end{frame}

	\begin{frame}
		\frametitle{Kategorien der Entwurfsmuster}
		\begin{itemize}
			\item \textbf{Entkopplungs-Muster}
			\begin{itemize}
				\item Adapter \colorbox{green}{fertig}
				\item Beobachter\colorbox{green}{fertig}
				\item Iterator \colorbox{green}{fertig}
				\item Stellvertreter \colorbox{green}{fertig}
				\item Vermittler \colorbox{green}{fertig}
				\item (Brücke)
			\end{itemize}
			\item Varianten-Muster
			\item Zustandshandhabungs-Muster
			\item Steuerungs-Muster
			\item Bequemlichkeits-Muster
		\end{itemize}
	\end{frame}

\section{Gruppenarbeit}
\subsection{Gruppenarbeit}
	\begin{frame}
		\frametitle{Kategorien der Entwurfsmuster}
		\begin{itemize}
			\item Entkopplungs-Muster \colorbox{green}{fertig}
			\item \textbf{Varianten-Muster}
			\begin{itemize}
				\item (Abstrakte Fabrik)
				\item (Besucher)
				\item \textbf{Schablonenmethode}
				\item \textbf{Fabrikmethode}
				\item \textbf{Kompositum}
				\item Strategie \colorbox{green}{fertig}
				\item \textbf{Dekorierer}
			\end{itemize}
			\item Zustandshandhabungs-Muster
			\item Steuerungs-Muster
			\item Bequemlichkeits-Muster
		\end{itemize}
	\end{frame}
	
	\begin{frame}
		\frametitle{Varianten-Muster}
		\begin{block}{Übergeordnetes Ziel}
			\begin{itemize}
				\item Gemeinsamkeiten herausziehen und an einer Stelle beschreiben \pause
				\linebreak $\implies$ keine Wiederholung desselben Codes \pause
				\linebreak $\implies$ bessere Wartbarkeit/Erweiterbarkeit		
			\end{itemize}
		\end{block}
	\end{frame}
	
	\begin{frame}
		\frametitle{Jetzt: Gruppenarbeit}
		\begin{enumerate}
			\item ihr kriegt pro Reihe eine Aufgabe
			\item ihr habt Zeit zum Bearbeiten
			\item Abgleichung mit Musterlösung
			\item ihr stellt den anderen eure Lösung vor
		\end{enumerate}
	\end{frame}

	\begin{frame}
		\frametitle{Vorstellung Dekorierer}
		\includegraphics[scale=0.35]{./pics/tut4/decor.png}
	\end{frame}

	\begin{frame}
		\frametitle{MuLö Dekorierer}
		\begin{block}{Wo Gemeinsamkeiten?}
			Die beiden Methoden methodeEins() und methodeZwei().
		\end{block}
		\begin{block}{Wo Variation?}
			In den KonkretenDekorierern bzw. ihren Methoden. Hier: neueMethodeA(), neueMethodeB().
		\end{block}
		\begin{block}{Wozu Instanzvariable?}
			Weiterleitung von Aufrufen der methodeEins() und methodeZwei() an die KonkreteKompenente.
		\end{block}
	\end{frame}

	\begin{frame}
		\frametitle{Vorstellung Kompositum}
		\includegraphics[scale=0.35]{./pics/tut4/comp.png}
	\end{frame}

	\begin{frame}
		\frametitle{MuLö Kompositum}
		\begin{block}{Wo Gemeinsamkeiten?}
			gemeinsameOperation().
		\end{block}
		\begin{block}{Wo Variation?}
			In Blatt/Kompositum-Klassen mit verschiedenen zusätzlichen Operationen.
		\end{block}
		\begin{block}{Zusammengesetzt vs. nicht-zusammengesetzt}
			Kompositum = zusammengesetzt, Blatt = nicht-zusammengesetzt
		\end{block}
	\end{frame}

	\begin{frame}
		\frametitle{Vorstellung Schablonenmethode}
		\includegraphics[scale=0.45]{./pics/tut4/schab.png}
	\end{frame}

	\begin{frame}
		\frametitle{MuLö Schablonenmethode}
		\begin{block}{Wo Gemeinsamkeiten?}
			Reihenfolge der Methodenaufrufe in der Schablonenmethode.
		\end{block}
		\begin{block}{Wo Variation?}
			In den Einschubmethoden. (hier: primitiveOperation1() und primitiveOpoeration2())
		\end{block}
		\begin{block}{Schablonenmethode vs. Einschubmethode}
			Einschubmethode ist eine der Methoden, die von der Schablonenmethode aufgerufen wird und deren Implementierung in den Unterklassen stattfindet.
		\end{block}
	\end{frame}

	\begin{frame}
		\frametitle{Vorstellung Fabrikmethode}
		\includegraphics[scale=0.4]{./pics/tut4/fab.png}
	\end{frame}

	\begin{frame}
		\frametitle{MuLö Fabrikmethode}
		\begin{block}{Wo Gemeinsamkeiten?}
			Reihenfolge der Methodenaufrufe in der beliebigenMethode().
		\end{block}
		\begin{block}{Wo Variation?}
			In der Fabrikmethode.
		\end{block}
		\begin{block}{Klasse des Objekts, Oberklasse, Unterklasse}
			Klasse des Objekts = KonkretesProdukt, Oberklasse = Produkt, Unterklasse = KonkreterErzeuger
		\end{block}
		\begin{block}{Unterschied zu Schablonenmethode?}
			Fabrikmethode benutzen, wenn ein Objekt erzeugt wird. Fabrikmethode ist Einschubmethode des Musters "'Schablonenmethode"'.
		\end{block}
		\begin{block}{Wahr/falsch}
			Fabrikmethode ist eine Einschubmethode, keine Schablonenmethode.
		\end{block}
	\end{frame}

	\section{Einzelstück}
	\subsection{Intro}
	\begin{frame}
		\frametitle{Kategorien der Entwurfsmuster}
		\begin{itemize}
			\item Entkopplungs-Muster \colorbox{green}{fertig}
			\item Varianten-Muster \colorbox{green}{fertig}
			\item \textbf{Zustandshandhabungs-Muster}
				\begin{itemize}
					\item \textbf{Einzelstück}
					\item (Fliegengewicht)
					\item \textbf{Memento} 
					\item (Prototyp) 
					\item (Zustand)
				\end{itemize}
			\item Steuerungs-Muster
			\item Bequemlichkeits-Muster
		\end{itemize}
	\end{frame}

	\begin{frame}
		\frametitle{Zustandshandhabungs-Muster}
		\begin{block}{Übergeordnetes Ziel}
			\begin{itemize}
				\item den Zustand eines Objektes beschreiben (wer hätt's gedacht? :D) \pause 
				\item aber unabhängig von dem Zweck des Objekts!
			\end{itemize}
		\end{block}
	\end{frame}

	\begin{frame}
		\frametitle{Einzelstück/Singleton}
		\begin{block}{Problem}
			\begin{itemize}
				\item von einer Klasse soll nur eine Instanz existieren
				\item Konstruktor könnte überall benutzt werden!
			\end{itemize}
		\end{block}
		\pause
		\centering
		\begin{figure}
			\includegraphics[scale=0.3]{./pics/tut4/singleton.png}
		\end{figure}
		\pause
		Aber warum nicht einfach statisch?\pause ~~ Unterklassenbildung möglich!
	\end{frame}

	\section{Memento}
	\subsection{Memento}
	\begin{frame}
		\frametitle{Memento}
		\begin{block}{Problem}
			\begin{itemize}
				\item internen Zustand eines Objekts "'externalisieren"', um z.B. Zurücksetzen möglich zu machen \pause 
				\item ohne Kapselung zu verletzten!
			\end{itemize}
		\end{block}
		\pause
		\centering
		\includegraphics[scale=0.4]{./pics/tut4/mem.png}
	\end{frame}

	\begin{frame}
		\frametitle{Memento}
		\includegraphics[scale=0.4]{./pics/tut4/mem.png}
		\begin{block}{Problem gelöst?}
			\begin{itemize}
				\pause
				\item Ja
				\begin{itemize}
					\pause
					\item Zustand durch Memento externalisiert \pause
					\item Kapselung nicht verletzt (Nutzer ruft nur sichereZustand() auf und kriegt neuen Memento)
				\end{itemize}
			\end{itemize}
		\end{block}
\end{frame}

\section{Befehl}
	\subsection{Intro}
	\begin{frame}
		\frametitle{Kategorien der Entwurfsmuster}
		\begin{itemize}
			\item Entkopplungs-Muster \colorbox{green}{fertig}
			\item Varianten-Muster \colorbox{green}{fertig}
			\item Zustandshandhabungs-Muster \colorbox{green}{fertig}
			\item \textbf{Steuerungs-Muster} 
				\begin{itemize}
					\item \textbf{Befehl}
					\item (master/worker)
				\end{itemize}
			\item Bequemlichkeits-Muster
		\end{itemize}
	\end{frame}
	
	\begin{frame}
		\frametitle{Steuerungs-Muster}
		\begin{block}{Übergeordnetes Ziel}
			\begin{itemize}
				\item steuern den Kontrollfluss \pause 
				\linebreak $\implies$ zur richtigen Zeit richtige Methoden aufrufen
			\end{itemize}
		\end{block}
	\end{frame}

	\begin{frame}
		\frametitle{Befehl}
		\begin{block}{Problem}
			\begin{itemize}
				\item Parametrisieren von Objekten mit einer auszuführenden Aktion \pause 
				\item komplexe Operationen aus primitiven Operationen aufbauen \pause
				\linebreak $\implies$ Befehl nicht als Methode, sondern als Objekt modellieren
			\end{itemize}
		\end{block}
		\pause
		\centering
		\includegraphics[scale=0.35]{./pics/tut4/command.png}
	\end{frame}

	\begin{frame}
		\frametitle{Befehl}
		\includegraphics[scale=0.35]{./pics/tut4/command.png}
		\begin{block}{Was haben wir erreicht?}
			\begin{itemize}
				 \pause
				\item Austauschbarkeit: Befehle unabhängig vom Aufrufer, universell einsetzbar
			\end{itemize}
		\end{block}
	\end{frame}
	
	\begin{frame}
		\frametitle{Befehl}
		\includegraphics[scale=0.4]{./pics/tut4/command.png}
		\linebreak
		\centering \large Beispiel!
	\end{frame}
	
	\begin{frame}
		\frametitle{Quiz (Ankreuzaufgaben aus Klausuren)}
		Wahr oder falsch?
		\begin{itemize}
			\item Bei dem Entwurfsmuster Befehl kennt der Empfänger den Befehl nicht, jedoch der Befehl den Empfänger. \pause \colorbox{green}{wahr} \pause
			\item Ein Aufbewahrer im Entwurfsmuster Memento kann beliebig viele Mementos verwalten. Für die Restauration im Falle eines Reset ist er allerdings nicht verantwortlich. \pause \colorbox{green}{wahr} \pause
			\item Die Fabrikmethode sorgt dafür, dass nur eine einzige Instanz einer Klasse fabriziert wird. \pause \colorbox{red}{falsch} \pause 
			\item Eine Schablonenmethode ist immer auch eine Fabrikmethode. \pause \colorbox{red}{falsch} \pause
			\item Eine Komponente kann immer nur mit einem einzigen Dekorierer versehen werden. \pause \colorbox{red}{falsch}
		\end{itemize}
	\end{frame}
	
	
	\begin{frame}
		\frametitle{Für die Klausur}
		\begin{itemize}
			\item Entwurfsmuster kommen sehr sehr sehr wahrscheinlich dran! \pause 
			\item Kategorien helfen beim Lernen \pause
			\item jedes Entwurfsmuster erfüllt einen bestimmten Zweck 
			\linebreak $\implies$ nicht nur die Klassen und Methoden auswendig lernen, sondern das Prinzip verstehen \pause
			\item bei Unklarheiten in Head First Design Patterns nachlesen ;)
		\end{itemize}
	\end{frame}
	
\section{Tipps}
	\subsection{Tipps}
	\begin{frame}
		\frametitle{Tipps - 5. Übungsblatt}
			\begin{exampleblock}{Aufgabe 1: Shutterpile: Refaktorisierung + Entwurfsmuster anwenden} 
				\begin{itemize}
					\item Entwurfsmuster anschauen
					\item alte Tests verwenden + evtl. neue schreiben
				\end{itemize}
			\end{exampleblock}
			\pause
			\begin{exampleblock}{Aufgabe 2: cmd-Programm für Pipeline} 
				\begin{itemize}
					\item wie Shutterpile-cmd, nur kommen nach Parameter \enquote{-p} noch Werte
					\item \url{https://commons.apache.org/proper/commons-cli/usage.html}
				\end{itemize}
			\end{exampleblock}
	\end{frame}

	\begin{frame}
		\frametitle{Tipps - 5. Übungsblatt}
			\begin{exampleblock}{Aufgabe 3: Wo sind Entwurfsmuster in Shutterpile?}
				\begin{itemize}
					\item Maßstab ist Musterlösung
					\item nur finden reicht nicht, auch erklären wie und warum
				\end{itemize}
			\end{exampleblock}
			\pause
			\begin{exampleblock}{Aufgabe 4: Entwurfsmuster in Java-API}
				\begin{itemize}
					\item es handelt sich um \enquote{einfachere} Muster
				\end{itemize}
			\end{exampleblock}
			\pause
			\begin{exampleblock}{Aufgabe 5: Entwurfsmuster - Kaffeemaschine}
				\begin{itemize}
					\item ein Muster anwenden
				\end{itemize}
			\end{exampleblock}
	\end{frame}
	
	\subsection{Abgabe}
	\begin{frame}
		\frametitle{Denkt dran!}
		\begin{alertblock}{Abgabe}
			\begin{itemize}
				\item Deadline am 27.6. um 12:00
				\item Aufgabe 3-5 handschriftlich
			\end{itemize}
		\end{alertblock}
	\end{frame}
		
	\begin{frame}
		\frametitle{Bis dann! (dann  := 03.07.18)}
		\centering
		\includegraphics[scale=0.4]{./comics/patterns.jpg}
	\end{frame}

\end{document}