\documentclass[18pt]{beamer}
\usepackage[utf8]{inputenc} % for the umlauts
\usepackage{subfigure}

\beamertemplatenavigationsymbolsempty
%% SLIDE FORMAT

% use 'beamerthemekit' for standard 4:3 ratio
% for widescreen slides (16:9), use 'beamerthemekitwide'

\usepackage{templates/beamerthemekit}
% \usepackage{templates/beamerthemekitwide}

\setcounter{tocdepth}{1}

%% TITLE PICTURE

% if a custom picture is to be used on the title page, copy it into the 'logos'
% directory, in the line below, replace 'mypicture' with the 
% filename (without extension) and uncomment the following line
% (picture proportions: 63 : 20 for standard, 169 : 40 for wide
% *.eps format if you use latex+dvips+ps2pdf, 
% *.jpg/*.png/*.pdf if you use pdflatex)

%\titleimage{mypicture}

%% TikZ INTEGRATION

% use these packages for PCM symbols and UML classes
% \usepackage{templates/tikzkit}
% \usepackage{templates/tikzuml}

% the presentation starts here

\usepackage{mathabx}
\usepackage{picture}
\usepackage[absolute,overlay]{textpos}
%\usepackage[texcoord,grid,gridunit=mm,gridcolor=red, subgridcolor=green]{eso-pic}
\setbeamercovered{invisible}
\setbeamertemplate{caption}{\raggedright\insertcaption\par}

\title[SWT1]{Softwaretechnik 1 - 4. Tutorium}
\subtitle{Tutorium 03}
\author{Felix Bachmann}
\date{26.06.2017}

\institute{KIT - Institut für Programmstrukturen und Datenorganisation (IPD)}

% Bibliography

\usepackage[citestyle=authoryear,bibstyle=numeric,hyperref,backend=biber]{biblatex}
\addbibresource{templates/example.bib}
\bibhang1em

\begin{document}

% change the following line to "ngerman" for German style date and logos
\selectlanguage{ngerman}

%title page
\begin{frame}
\titlepage
\end{frame}

\begin{frame}
\tableofcontents
\end{frame}


\section{Orga}

	\subsection{Allgemein}
	\begin{frame}
		\frametitle{Allgemeines}
		\begin{alertblock}{Ansage der Übungsleiter}
			\begin{itemize}
				\item ab jetzt keine Abgabe per Mail mehr!
				\linebreak $\implies$ auch nicht in Ausnahmefällen
			\end{itemize}
		\end{alertblock}
	\end{frame}

	\subsection{Feedback 4. Übungsblatt}
	\begin{frame}
		\frametitle{4. Übungsblatt Statistik}
		%TODO statistics\includegraphics[scale=0.7]{./pics/tut3/statistics-ub3.png}
		%TODO avg \linebreak \centering $\diameter$ 
	\end{frame}
	
	\subsection{4. Übungsblatt - Fehler (Allgemein)}
	\begin{frame}
		\frametitle{Häufige Fehler}
		\begin{block}{Allgemein}
			\begin{itemize}
				\item %TODO 
			\end{itemize}
		\end{block}
	\end{frame}
	
	\subsection{4. Übungsblatt - Fehler}
	\begin{frame}
		\frametitle{Häufige Fehler}
		\begin{block}{Aufgabe 1 (Zustandsdiagramm - LEZ): $\diameter$}%TODO avg
			\begin{itemize}
				\item %TODO
			\end{itemize}
		\end{block}
	\end{frame}

	\begin{frame}
		\frametitle{Häufige Fehler}
		\begin{block}{Aufgabe 2 (Abbottsche Methode): $\diameter$}%TODO avg
			\begin{itemize}
				\item %TODO
			\end{itemize}
		\end{block}
		\begin{block}{Aufgabe 3 (iMage-GUI): $\diameter$}%TODO avg
			\begin{itemize}
				\item	%TODO
			\end{itemize}
		\end{block}
	\end{frame}

	\begin{frame}
		\frametitle{Häufige Fehler}
		\begin{block}{Aufgabe 4 (Geheimnisprinzip): $\diameter$}%TODO avg
			\begin{itemize}
				\item %TODO
			\end{itemize}
		\end{block}
	\end{frame}

\section{Recap}
	\subsection{Quiz(Adapter)}
	\begin{frame}
		\frametitle{Was bisher geschah..}
		\begin{itemize}
			\item haben uns erste Entkopplungmuster angeschaut \pause
			\linebreak $\implies$ Beobachter, Iterator, Adapter
		\end{itemize}
		\begin{figure}
			\includegraphics[scale=0.33]{./pics/tut4/adap-obj-mod.png}
		\end{figure}
		Welches Entwurfsmuster? \pause (Objekt-)Adapter
	\end{frame}
	
	\begin{frame}
		\frametitle{Was bisher geschah..}
		\begin{itemize}
			\item haben uns erste Entkopplungmuster angeschaut
			\linebreak $\implies$ Beobachter, Iterator, Adapter
		\end{itemize}
		\begin{figure}
			\includegraphics[scale=0.33]{./pics/tut4/adap-obj-mod.png}
		\end{figure}
		Welche Klassen?
	\end{frame}
	
	\begin{frame}
		\frametitle{Was bisher geschah..}
		\begin{itemize}
			\item haben uns erste Entkopplungmuster angeschaut
			\linebreak $\implies$ Beobachter, Iterator, Adapter
		\end{itemize}
		\begin{figure}
			\includegraphics[scale=0.45]{./pics/tut3/adap-obj.png}
		\end{figure}
	\end{frame}

\section{Stellvertreter}
	\subsection{Stellvertreter}
	\begin{frame}
		\frametitle{Stellvertreter}
		\begin{block}{Problem}
			\begin{itemize}
				\item wollen Zugriff auf ein Objekt kontrollieren, ohne seine Klasse zu ändern \linebreak \pause $\implies$ Stellvertreter macht Zugriffskontrolle
			\end{itemize}
		\end{block}
		\pause
		\centering
		\includegraphics[scale=0.4]{./pics/tut3/prox.png}
	\end{frame}

	\begin{frame}
		\frametitle{Stellvertreter}
		\centering
		\includegraphics[scale=0.4]{./pics/tut3/prox.png}
		\begin{block}{Entkopplung?}
			\begin{itemize}
				\pause 
				\item Klient hat keinen direkten Zugriff auf das echte Subjekt
			\end{itemize}
		\end{block}
	\end{frame}

\section{Vermittler}
	\subsection{Vermittler}
		\begin{frame}
		\frametitle{Vermittler}
		\begin{block}{Problem}
			\begin{itemize}
				\item mehrere voneinander abhängige Objekte \linebreak \pause $\implies$ Zustände der Objekte von anderen Zuständen abhängig
			\end{itemize}
		\end{block}
		\pause
		\centering
		\includegraphics[scale=0.45]{./pics/tut3/med.png}
	\end{frame}

	\begin{frame}
		\frametitle{Vermittler}
		\centering
		\includegraphics[scale=0.45]{./pics/tut3/med.png}
		\begin{block}{Entkopplung?}
			\begin{itemize}
				\pause 
				\item Kollegen kennen sich nicht direkt  \linebreak \pause $\implies$ Hinzufügen eines Kollegen erfordert keine Änderung der alten Kollegen
			\end{itemize}
		\end{block}
	\end{frame}



\section{Tipps}
	\subsection{Tipps}
	\begin{frame}
		\frametitle{Tipps - 5. Übungsblatt}
			\begin{exampleblock}{Aufgabe 1: } %TODO
				\begin{itemize}
					\item %TODO
				\end{itemize}
			\end{exampleblock}
			\pause
			\begin{exampleblock}{Aufgabe 2: } %TODO
				\begin{itemize}
					\item %TODO
				\end{itemize}
			\end{exampleblock}
	\end{frame}

	\begin{frame}
		\frametitle{Tipps - 5. Übungsblatt}
			\begin{exampleblock}{Aufgabe 3:} %TODO
				\begin{itemize}
					\item %TODO
				\end{itemize}
			\end{exampleblock}
			\pause
			\begin{exampleblock}{Aufgabe 4:} %TODO
				\begin{itemize}
					\item %TODO
				\end{itemize}
			\end{exampleblock}
	\end{frame}
	
	\subsection{Abgabe}
	\begin{frame}
		\frametitle{Denkt dran!}
		\begin{alertblock}{Abgabe}
			\begin{itemize}
				\item Deadline am 5.7. um 12:00 %TODO check when released
				%TODO self-written,...?
			\end{itemize}
		\end{alertblock}
	\end{frame}
		
	\begin{frame}
		\frametitle{Bis dann! (dann  := 10.07.17)}
		\centering
		%TODO choose comic \includegraphics[scale=0.5]{./comics/geek_and_poke_undeadCode.jpg}
	\end{frame}

\end{document}