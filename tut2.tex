\documentclass[18pt]{beamer}
\usepackage[utf8]{inputenc} % for the umlauts
\usepackage{subfigure}

\beamertemplatenavigationsymbolsempty
%% SLIDE FORMAT

% use 'beamerthemekit' for standard 4:3 ratio
% for widescreen slides (16:9), use 'beamerthemekitwide'

\usepackage{templates/beamerthemekit}
% \usepackage{templates/beamerthemekitwide}

\setcounter{tocdepth}{1}

%% TITLE PICTURE

% if a custom picture is to be used on the title page, copy it into the 'logos'
% directory, in the line below, replace 'mypicture' with the 
% filename (without extension) and uncomment the following line
% (picture proportions: 63 : 20 for standard, 169 : 40 for wide
% *.eps format if you use latex+dvips+ps2pdf, 
% *.jpg/*.png/*.pdf if you use pdflatex)

%\titleimage{mypicture}

%% TikZ INTEGRATION

% use these packages for PCM symbols and UML classes
% \usepackage{templates/tikzkit}
% \usepackage{templates/tikzuml}

% the presentation starts here

\usepackage{mathabx}
\usepackage{picture}
\usepackage[absolute,overlay]{textpos}
%\usepackage[texcoord,grid,gridunit=mm,gridcolor=red, subgridcolor=green]{eso-pic}
\setbeamercovered{invisible}
\setbeamertemplate{caption}{\raggedright\insertcaption\par}

\title[SWT1]{Softwaretechnik 1 - 2. Tutorium}
\subtitle{Tutorium 03}
\author{Felix Bachmann}
\date{29.05.2017}

\institute{KIT - Institut für Programmstrukturen und Datenorganisation (IPD)}

% Bibliography

\usepackage[citestyle=authoryear,bibstyle=numeric,hyperref,backend=biber]{biblatex}
\addbibresource{templates/example.bib}
\bibhang1em

\begin{document}

% change the following line to "ngerman" for German style date and logos
\selectlanguage{ngerman}

%title page
\begin{frame}
\titlepage
\end{frame}

\section{Orga}
	\subsection{Feedback 2. Übungsblatt}
	\begin{frame}
		\frametitle{2. Übungsblatt Statistik}
		%TODO add statistics chart [scale=0.7]{./pics/tut2/statistics_ub1.png}
	\end{frame}
	
	\subsection{2. Übungsblatt - Fehler (Allgemein)}
	\begin{frame}
		\frametitle{Häufige Fehler}
		\begin{block}{Allgemein}
			\includegraphics[scale=0.6]{./pics/tut2/deckblatt.png}
			\begin{itemize}
				\item $\implies$ nur das offizielle Deckblatt verwenden!
				\pause
				\item häufigster Fehler: Aufgaben nicht abgegeben
			\end{itemize}
		\end{block}
	\end{frame}
	
	\subsection{2. Übungsblatt - Fehler}
	\begin{frame}
		\frametitle{Häufige Fehler}
		\begin{block}{Aufgabe 1 (Lastenheft): $\diameter$ 2,29 bzw. 3,44 von 5}
			\begin{itemize}
				\item Unnötiges aus Vorlage durfte man löschen (z.B. "'Siehe https://en.wikibooks.org/wiki/LaTeX/Glossary"' oder "'Szenarien"')
				\pause
				\item Anwendungsfälle beschreiben, was man in dem System tun kann 
				\linebreak $\implies$ beinhalten immer Verben! (z.B. "'Hobbyfotograph"' oder "'JMJRST"' ist kein Anwendungsfall)
			\end{itemize}
		\end{block}
	\end{frame}

	\begin{frame}
		\frametitle{Häufige Fehler}
		\begin{block}{Aufgabe 2 (Klassendiagramm): $\diameter$ 2,54 bzw. 4,07 von 8}
			\begin{itemize}
				\item oft wurden abstrakte Klassen "'Maler-Filter"' und "'Kunst-Filter"' vergessen
				\pause
				\item es wurden Dinge ergänzt, die so nicht explizit im Text standen
				\pause
				\item Ausrichtung des Bildes wurde nicht als Enum modelliert
				\pause
				\item falsche UML-Syntax (insb. Methode, Attribute)
			\end{itemize}
		\end{block}
	\end{frame}

	\begin{frame}
		\frametitle{Häufige Fehler}
		\begin{block}{Aufgabe 3 (Durchführbarkeitsanalyse): $\diameter$ 0,81 bzw. 1,95 von 3}
			\begin{itemize}
				\item sehr häufig nicht abgegeben
				\pause
				\item Fragen beantworten, nicht stellen!
				\linebreak z.B. "'Es werden 3 Java-Entwickler benötigt."' ergänzen durch
				"'Da wir 5 zur Zeit untätige Java-Entwickler in der Firma haben, ist das Projekt aus personeller Sicht für die Pear Corp. durchführbar."'
			\end{itemize}
		\end{block}
	\end{frame}


\section{Zustandsdiagramm}
	\subsection{Intro(1)}
	\begin{frame}
		\frametitle{Wo sind wir? Pflichtenheft!}
		\begin{enumerate}
			\item Zielbestimmung  
			\item Produkteinsatz 
			\item Produktumgebung
			\item Funktionale Anforderungen 
			\item Produktdaten 
			\item Nichtfunktionale Anforderungen 
			\item Globale Testfälle
			\item Systemmodelle
			\begin{itemize}
				\item Szenarien
				\item Anwendungsfälle
				\item Objektmodelle $\implies$ UML-Klassendiagramme (letztes mal)
				\item \underline{\textbf{Dynamische Modelle}}
				\begin{itemize}
					\item UML-Zustandsdiagramm
					\item UML-Aktivitätsdiagramm
					\item UML-Sequenzdiagramm
					\makebox(0,0){\put(0,3.3\normalbaselineskip){%
							$\left.\rule{0pt}{1.5\normalbaselineskip}\right\}$ Heute!}}
				\end{itemize}
				\item Benutzerschnittstelle$\implies$ Zeichnungen/Screenshots
			\end{itemize}
			\item Glossar 
		\end{enumerate}
	\end{frame}

	\subsection{Intro(2)}
	\begin{frame}
		\frametitle{Begriffsklärung}
		\includegraphics[scale=0.35]{./pics/tut1/uml_diagrams.png}
	\end{frame}

	\subsection{ZD: Allgemein}
	\begin{frame}
		\frametitle{Zustandsdiagramm - Allgemein}
		\begin{block}{Wozu braucht man das?}
			\pause
			\begin{itemize}
				\item Zustand \textbf{eines Objektes} beschreiben
				\item Zustandsüberführungsfunktion?
			\end{itemize}
		\end{block}
	\end{frame}

	\subsection{ZD: Allgemein (2)}
	\begin{frame}
		\frametitle{Zustandsdiagramm $\approx$ endlicher Automat}
		\begin{figure}
			\subfigure[GBI: DEA]{\includegraphics[scale=0.23]{./pics/tut2/auto_gbi.png}}
			\subfigure[SWT: Zustandsdiagramm]{\includegraphics[scale=0.2]{./pics/tut2/auto_swt.png}}
		\end{figure}
	\end{frame}

	\subsection{ZD: Syntax(1)}
	\begin{frame}
		\frametitle{Zustandsdiagramm: Syntax}
		\begin{figure}
			\includegraphics[scale=0.4]{./pics/tut2/auto_swt.png}
		\end{figure}	
	\end{frame}

	\subsection{ZD: Syntax (2)}
	\begin{frame}
		\frametitle{Zustandsdiagramm: Hierarchie}
		\includegraphics[scale=0.7]{./pics/tut2/auto_hier.png}	
	\end{frame}

	\subsection{ZD: Syntax (3)}
	\begin{frame}
		\frametitle{Zustandsdiagramm: Hierarchie - History}
		\begin{itemize}
			\item  History-Element, damit sich Hierarchie den letzten Zustand merkt
		\end{itemize}
		\includegraphics[scale=0.5]{./pics/tut2/auto_hier-hist.png}	
	\end{frame}

	\subsection{ZD: Syntax(4)}
	\begin{frame}
		\frametitle{Zustandsdiagramm: Nebenläufigkeit}	
		\begin{itemize}
			\item  mehrere Zustandsdiagramme in einem
		\end{itemize}
		\centering
		\includegraphics[scale=0.5]{./pics/tut2/auto_par.png}	
	\end{frame}

	\subsection{ZD: Aufgabe}
	\begin{frame}
		\frametitle{Klausuraufgabe SS09}
		Gegeben ist der folgende UML-Zustandsautomat. Geben Sie an, in welcher Zustandskombination
		sich der Zustandsautomat, jeweils ausgehend vom Startzustand, nach den beiden Eingabefolgen
		befindet.
		\centering
		\includegraphics[scale=0.7]{./pics/tut2/auto_ex.png}
		\begin{itemize}
			\item a, b, c, c \pause $\implies$ AxD
			\item c, c, a, b, b, a, c, c, a \pause $\implies$ BxC
		\end{itemize}
	\end{frame}
		
\section{Aktivitätsdiagramm}
	\subsection{AD: Allgemein(1)}
	\begin{frame}
		\frametitle{Aktivitätsdiagramm - Allgemein}
		\begin{block}{Wozu braucht man das?}
			\pause
			\begin{itemize}
				\item Ablaufbeschreibungen (Kontrollfluss, Objektfluss)
				\item i.A. \textbf{mehrere verschiedene} Objekte
			\end{itemize}
		\end{block}
	\end{frame}

	\subsection{AD: Allgemein(2)}
	\begin{frame}
		\frametitle{Aktivitätsdiagramm - Beispiel}
		\begin{itemize}
			\item ist ebenfalls nicht neues!
		\end{itemize}
		\centering
		\includegraphics[scale=0.35]{./pics/tut2/act_wat.png}
	\end{frame}

	\subsection{AD: Syntax(1)}
	\begin{frame}
		\frametitle{Aktivitätsdiagramm - Syntax}
		\includegraphics[scale=0.45]{./pics/tut2/act_syn1.png}
	\end{frame}

	\subsection{AD: Syntax(2)}
	\begin{frame}
		\frametitle{Aktivitätsdiagramm - Syntax}
		\includegraphics[scale=0.45]{./pics/tut2/act_syn2.png}
	\end{frame}

	\subsection{AD: Syntax(3)}
	\begin{frame}
		\frametitle{Aktivitätsdiagramm - Syntax}
		\includegraphics[scale=0.45]{./pics/tut2/act_syn3.png}
	\end{frame}

	\subsection{AD: Ablauf}
	\begin{frame}
		\frametitle{Aktivitätsdiagramm - Ablauf}
		\begin{itemize}
			\item Start am Startknoten mit einer Marke
			\item Aktivitäten werden erst ausgeführt, wenn an jedem Eingang eine Marke anliegt
			\item wurde eine Aktivität ausgeführt, erscheinen an all ihren Ausgängen Marken
		\end{itemize}
	\end{frame}

	\subsection{AD: Beispiel}
	\begin{frame}
		\frametitle{Aktivitätsdiagramm - Beispiel}
		\begin{figure}
			\centering
			\caption{Wie kommt man hier zum Endknoten?}
			\includegraphics[scale=0.4]{./pics/tut2/act_ex.png}
		\end{figure}
	\end{frame}

\section{Sequenzdiagramm}
	\subsection{SD: Allgemein}
	\begin{frame}
		\frametitle{Sequenzdiagramm - Allgemein}
		\begin{block}{Wozu braucht man das?}
			\pause
			\begin{itemize}
				\item stellt den möglichen Ablauf eines Anwendungsfalls dar
				\item \textbf{zeitlicher Verlauf} von Methodenaufrufen, Objekterstellung, Objektzerstörung
			\end{itemize}
		\end{block}
	\end{frame}

	\subsection{SD: Syntax(1)}
	\begin{frame}
		\frametitle{Sequenzdiagramm - Syntax}
		\begin{itemize}
			\item Zeit verläuft von oben nach unten
			\item Lebenslinie
			\begin{itemize}
				\item gestrichelte senkrechte Linie
				\item eine pro Objekt
			\end{itemize}
			\item Steuerungsfokus
			\begin{itemize}
				\item dicker Balken über Lebenslinie
				\item zeigt, dass Objekt gerade aktiv ist
			\end{itemize}
			\item Nachrichtentypen
			\begin{itemize}
				\item Synchrone Nachricht (blockierend)
				\item Antwort (optional)
				\item Asynchrone Nachricht
			\end{itemize}
			\begin{textblock*}{20mm}(75mm,60mm)
				\includegraphics[scale=0.4]{./pics/tut2/sd_met.png}
			\end{textblock*}
		\end{itemize}
	\end{frame}

	\subsection{SD: Syntax(2)}
	\begin{frame}
		\frametitle{Sequenzdiagramm - Syntax}
			\includegraphics[scale=0.4]{./pics/tut2/sd_ex.png}
	\end{frame}

	\subsection{SD: Syntax(2)}
	\begin{frame}
		\frametitle{Sequenzdiagramm - Syntax}
		\begin{figure}
			\centering
			\subfigure[Objekt-Erzeugung]{\includegraphics[scale=0.35]{./pics/tut2/sdcrea.png}}\newline
			\subfigure[Objekt-Zerstörung]{\includegraphics[scale=0.2]{./pics/tut2/sddestr.png}}
		\end{figure}
	\end{frame}

	\subsection{SD: Aufgabe}
	\begin{frame}
		\frametitle{Klausuraufgabe SS14}
			\begin{figure}
				\centering
				\includegraphics[scale=0.65]{./pics/tut2/sdtask.png}
				\caption{Hier stimmt was nicht\dots}
			\end{figure}
	\end{frame}

%TODO Methode: Gruppenarbeit

\section{Tipps}
	\subsection{Tipps}
	\begin{frame}
		\frametitle{Tipps - 3. Übungsblatt}
			\begin{exampleblock}{Aufgabe 1-3: Plug-In programmieren}
				\begin{itemize}
					\pause
					\item JavaDoc + CheckStyle \dots
					\item Falls ihr Tests schreibt, fügt junit in die jeweilige Untermodul-pom ein
					\item Java Swing benutzen (schaut euch die Java-Klassen JMenu und JMenuItem an)
				\end{itemize}
			\end{exampleblock}
			\pause
			\begin{exampleblock}{Aufgabe 4: Aktivitätsdiagramm}
				\begin{itemize}
					\pause 
					\item seperate Diagramme $\implies$ verschachtelte Aktionen
				\end{itemize}
			\end{exampleblock}
	\end{frame}

	\begin{frame}
		\frametitle{Tipps - 3. Übungsblatt}
			\begin{exampleblock}{Aufgabe 5: Sequenzdiagramm}
				\begin{itemize}
					\pause
					\item auf welchen Objekten/Klassen werden Methoden aufgerufen?
					\item auf Pfeile var=methode() schreiben, wenn Rückgabe von methode() in var gespeichert wird
				\end{itemize}
			\end{exampleblock}
			\pause
			\begin{exampleblock}{Aufgabe 6: Substitutionsprinzip}
				\begin{itemize}
					\pause
					\item Folien "'Folgerung aus dem Substitutionsprinzip"' anschauen (Ko-/Kontravarianz)
					\item mal als Java-Programm hinschreiben und versuchen zu kompilieren
				\end{itemize}
			\end{exampleblock}
	\end{frame}
	
	\subsection{Abgabe}
	\begin{frame}
		\frametitle{Denkt dran!}
		\begin{alertblock}{Abgabe}
			\begin{itemize}
				\item Deadline am 7.6 um 12:00
				\item Aufgabe 4+5 handschriftlich (auf saubere Syntax achten!)
				\item an das Deckblatt denken!!
			\end{itemize}
		\end{alertblock}
	\end{frame}
		
	\begin{frame}
		\frametitle{Bis dann! (dann := 12.06.17)}
		\centering
		\includegraphics[scale=0.9]{./comics/geek_and_poke_development.jpg}
	\end{frame}

\end{document}
