\documentclass[18pt]{beamer}
\usepackage[utf8]{inputenc} % for the umlauts
\usepackage{subfigure}

\beamertemplatenavigationsymbolsempty
%% SLIDE FORMAT

% use 'beamerthemekit' for standard 4:3 ratio
% for widescreen slides (16:9), use 'beamerthemekitwide'

\usepackage{templates/beamerthemekit}
% \usepackage{templates/beamerthemekitwide}

\setcounter{tocdepth}{1}

%% TITLE PICTURE

% if a custom picture is to be used on the title page, copy it into the 'logos'
% directory, in the line below, replace 'mypicture' with the 
% filename (without extension) and uncomment the following line
% (picture proportions: 63 : 20 for standard, 169 : 40 for wide
% *.eps format if you use latex+dvips+ps2pdf, 
% *.jpg/*.png/*.pdf if you use pdflatex)

%\titleimage{mypicture}

%% TikZ INTEGRATION

% use these packages for PCM symbols and UML classes
% \usepackage{templates/tikzkit}
% \usepackage{templates/tikzuml}

% the presentation starts here

\usepackage{mathabx}
\usepackage{picture}
\usepackage[absolute,overlay]{textpos}
%\usepackage[texcoord,grid,gridunit=mm,gridcolor=red, subgridcolor=green]{eso-pic}
\setbeamercovered{invisible}
\setbeamertemplate{caption}{\raggedright\insertcaption\par}

\title[SWT1]{Softwaretechnik 1 - 6. Tutorium}
\subtitle{Tutorium 18}
\author{Felix Bachmann}
\date{17.08.2018}
\institute{KIT - Institut für Programmstrukturen und Datenorganisation (IPD)}

% Bibliography

\usepackage[citestyle=authoryear,bibstyle=numeric,hyperref,backend=biber]{biblatex}
\addbibresource{templates/example.bib}
\bibhang1em

\begin{document}
	
% change the following line to "ngerman" for German style date and logos
\selectlanguage{ngerman}
\setcounter{tocdepth}{2}
	
%title page
\begin{frame}
\titlepage
\end{frame}

\begin{frame}
\tableofcontents
\end{frame}


\section{Orga}
	\begin{frame}
		\frametitle{Allgemein}
		\begin{block}{Klausur, Übungsschein}
			\begin{itemize}
				\item Hauptklausur am 26.07.18, 15:00
				\item Nachklausur wahrscheinlich am 08.10.18
				\item Anmeldung sollte nun für alle möglich sein
			\end{itemize}
		\end{block}
	\end{frame}

	\subsection{Feedback}
	\begin{frame}
		\frametitle{6. Übungsblatt Statistik}
		%TODO
		\includegraphics[scale=0.7]{./pics/tut6/statistics-ub6.png}
		\linebreak \centering $\diameter$ 6,1 bzw. 13,3 von 26+4
	\end{frame}

	\begin{frame}[fragile]
		\frametitle{Häufige Fehler}
		\begin{block}{Aufgabe 1: Kontrollfluss-orientiertes Testen}
			\begin{itemize}
				\pause 
				\item alles außer Kontrollfluss-Zeug so lassen wie es ist! \pause
				\item Kontrollfluss-Zeug, das nicht if(x) goto ist auflösen! \pause
				\item außerdem Kurzschlussauswertung in zwei if auflösen 
			\end{itemize}
		\end{block}
		\begin{verbatim}
		if(x && y) {
		  z++;
		}
		
		if(x || y) {
		  z++;
		}
		\end{verbatim}
\end{frame}

	\begin{frame}
		\frametitle{KFO: Kurzschlussauswertung}
		\begin{itemize}
			\item Kurzschlussauswertung (\&\& bzw. $\Vert$) muss berücksichtigt werden \pause
			\item Erinnerung: 
			\begin{itemize}
				\item \&\& und $\Vert$ werten die rechte Seite nur aus, wenn notwendig
				\item \& und $\vert$ werten immer beide Seiten aus \pause
			\end{itemize}
		\end{itemize}
		\centering \includegraphics[scale=0.4]{./pics/tut6/code.png}
	\end{frame}

	\begin{frame}
		\frametitle{KFO: Kurzschlussauswertung}
		\begin{figure}
			\subfigure[1. Zwischensprache]{\includegraphics[scale=0.35]{./pics/tut6/zwischen.png}}	
			\subfigure[KFG]{\includegraphics[scale=0.35]{./pics/tut6/kfg.png}}
		\end{figure}
	\end{frame}

	\begin{frame}
		\frametitle{Häufige Fehler}
		\begin{block}{Aufgabe 2: Parallelisierung}
			\begin{itemize}
				\item 5 Abgaben, meist richtig \pause
				\item Anzahl Prozessoren berechnen
				\begin{itemize}
					\item Runtime.getRuntime().availableProcessors();
				\end{itemize}
			\end{itemize}
			\end{block}
		\pause 
		\begin{block}{Aufgabe 3: Abnahmetests}
			\begin{itemize}
				\item 4 Abgaben
				\item Test brauchen immer Asserts!
			\end{itemize}
		\end{block}
	\end{frame}

	\begin{frame}
		\frametitle{Häufige Fehler}
		\begin{block}{Aufgabe 4: Wettbewerb}
			\begin{itemize}
				\item 3 Abgaben
			\end{itemize}
		\end{block}
	\end{frame}
		
\section{Testen}
	\subsection{Definitionen}
	
	\begin{frame}
		\frametitle{Arten von Fehlern}
		\begin{itemize}
			\item Was verursacht was?
			\item Defekt, Irrtum, Versagen \pause
			\item "'Testing shows the presence of bugs, not their absence."' (Edsger W. Dijkstra)
		\end{itemize}
	\end{frame}

	\begin{frame}
		\frametitle{Arten von Tests}
		\begin{block}{Dynamische Verfahren}
			\begin{itemize}
				\item Testfälle schreiben und ausführen (z.B. mit JUnit) \pause
				\item white box testing \pause
				\begin{itemize}
					\item kontrollflussorientiert
					\item datenflussorientiert
				\end{itemize}
				\item black box testings \pause
				\begin{itemize}
					\item funktionale Tests \pause
					\item Leistungstests
				\end{itemize}
			\end{itemize}
		\end{block}
		\pause
		\begin{block}{Statische Verfahren}
			\begin{itemize}
				\item Inspektion \pause
				\item statische Analyse mit Tools \pause
				\item Programm wird nicht ausgeführt!
			\end{itemize}
		\end{block}
	\end{frame}


	\begin{frame}
		\frametitle{KFO: Klausuraufgabe mit Kurzschlussauswertung}
		\begin{huge}
			\centering Hauptklausur SS2016 A6
		\end{huge}
	\end{frame}

\section{Wiederholung und Klausuraufgaben}

	\subsection{Planung \& Definition}
	\begin{frame}
		\frametitle{Disclaimer}
		\begin{large}
			\begin{itemize}
				\item Ich kenne die Klausur auch nicht! \pause
				\linebreak $\implies$ alles, was ich zum Inhalt der Klausur sage ist Spekulation
				\begin{itemize}
					\item basierend auf Altklausuren \pause
				\end{itemize}
				\item kein Anspruch auf Vollständigkeit der Wiederholung
			\end{itemize}
		\end{large}
	\end{frame}

	\begin{frame}
		\frametitle{Die typische SWT-Klausur}
		\begin{enumerate}
			\item Aufgabe 1: Wahr-/Falsch-Fragen (ein paar gesammelt auf \url{www.github.com/malluce/swt1-tut})
			 und Wissensfragen\pause
			\item meistens Aufgaben zu:
			\begin{itemize}
				\item UML-Diagrammen \pause
				\item Entwurfsmustern \pause
				\item Parallelität \pause
				\item Testen bzw. Qualitätssicherung \pause
				\item Rest (z.B. Objektorientierung, Abbott, Prozessmodelle\dots) \pause
			\end{itemize}
		\end{enumerate}
		\begin{itemize}
			\item 1/3$\pm \epsilon$ der Punkte reichen (meistens) zum Bestehen
		\end{itemize}
	\end{frame}

	\begin{frame}
		\frametitle{Aufwärmaufgabe}
		\begin{huge}
			\centering Hauptklausur SS2011 A1
		\end{huge}
	\end{frame}

	\begin{frame}
		\frametitle{Planung \& Definition}
		\begin{itemize}
			\item Lastenheft, Pflichtenheft \pause
			\begin{itemize}
				\item Phasen zuordnen \pause
				\item Gliederung kennen \pause
				\item Beispiele geben \pause
			\end{itemize}
			\item UML-Diagramme \pause
			\begin{itemize}
				\item Klassendiagramm \pause
				\item Aktivitäts-, Sequenz-, Zustandsdiagramm \pause 
				\item Anwendungsfalldiagramm \pause
				\item Syntax kennen! \pause
				\item gegebenen Text in Diagramm umwandeln \pause
				\item bei Zustandsdiagramm
				\begin{itemize}
					\item Umwandeln hierarchisch $\Leftrightarrow$ nicht-hierarchisch 
					\item Umwandeln parallel $\Leftrightarrow$ nicht-parallel
				\end{itemize}
			\end{itemize}
		\end{itemize}
	\end{frame}

	\begin{frame}
		\frametitle{Klassendiagramm-Aufgabe}
		\begin{huge}
				\centering Nachklausur SS2011 A5
		\end{huge}
	\end{frame}

	\begin{frame}
		\frametitle{Zustandsdiagramm-Aufgabe}
		\begin{huge}
			 	\centering Hauptklausur SS2012 A2
		\end{huge}
	\end{frame}

	\subsection{Entwurf}
	\begin{frame}
		\frametitle{Entwurf}
		\begin{itemize}
			\item Architekturstile \pause
			\item \textbf{Entwurfsmuster} \pause
			\begin{itemize}
				\item möglichst viele, bestenfalls alle kennen und verstehen \pause
				\item Kategorien kennen \pause
				\item Klassendiagramm hinzeichnen \pause
				\item aus Klassendiagrammen Entwurfsmuster erkennen \pause
				\item Code für einfache Muster (Singleton\dots) schreiben \pause
				\item Code-Schnipsel auf mögliche Verbesserung durch EM untersuchen
			\end{itemize}
		\end{itemize}
	\end{frame}

	\begin{frame}
		\frametitle{Entwurfsmuster-Aufgabe}
		\begin{huge}
				\centering Hauptklausur SS2010 A3
		\end{huge}
	\end{frame}

	
	\subsection{Implementierung}
	\begin{frame}
		\frametitle{Implementierung}
		\begin{itemize}
			\item UML-Abbildung \pause
			\item \textbf{Parallelität} \pause
			\begin{itemize}
				\item grundlegendes Prinzip \pause
				\item in Java \pause
				\item critical sections/ race conditions \pause
				\item deadlock \pause
				\item Monitore, wait \& notify \pause
				\item Semaphore \pause
			\end{itemize}
			\item Rechnungen können (Speedup, Amdahls Law, \dots) \pause
			\item gegebenen Code thread-safe machen \pause
			\item Lösungsvorschläge zur Synchronisation bewerten \pause
			\item eigenen Code schreiben
		\end{itemize}
	\end{frame}

	\begin{frame}
		\frametitle{Parallelität-Aufgabe}
		\begin{huge}
				\centering Hauptklausur SS2011 A3
		\end{huge}
	\end{frame}
	
	\subsection{Testen}
	\begin{frame}
		\frametitle{Testen}
		\begin{itemize}
			\item Testphasen \pause
			\item Testverfahren
			\begin{itemize}
				\item \textbf{KFO} \pause
			\end{itemize}
			\item Testhelfer \pause
			\item Definitionen kennen (Fehlerarten\dots) \pause
			\item KFO im Schlaf können ("'Schema F"', lässt sich sehr gut üben\dots)
		\end{itemize}
	\end{frame}
	
	\subsection{Abnahme, Einsatz \& Wartung}
	\begin{frame}
		\frametitle{Abnahme, Einsatz \& Wartung}
		\begin{itemize}
			\item Aufgaben der verschiedenen "'Subphasen"' kennen \pause
			\item viel Text zum Lernen, aber nicht schwierig\dots \pause
			\item Wartung vs. Pflege \pause
			\item wahrscheinlich Ankreuzaufgaben dazu
		\end{itemize}
	\end{frame}

	\subsection{Rest}
	\begin{frame}
		\frametitle{Rest}
		\begin{itemize}
			\item Schätzmethoden \pause
			\item Prozessmodelle \pause
			\item Agile Prozesse \pause
			\item verschiedene Modelle kennen (XP, Scrum,\dots) \pause
			\item auch eher Ankreuzaufgaben, Wissensfragen
		\end{itemize}
	\end{frame}

\section{Ende}
	\subsection{Feierabend}
	\begin{frame}
		\frametitle{Lernen}
		\begin{itemize}
			\item Klausuren rechnen \&\& Folien anschauen
		\end{itemize}
	\end{frame}

	\begin{frame}
		\frametitle{Das war's dann wohl\dots}
		\centering
		\begin{large}
			Viel Erfolg bei der Klausur und im weiteren Studium! :)
		\end{large}
		\linebreak
		\includegraphics[scale=0.75]{./comics/geek_and_poke_development2.jpg}
	\end{frame}

\end{document}